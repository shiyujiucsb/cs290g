\documentclass[12pt]{article}
\usepackage{url,amsmath,setspace,amssymb,amsthm,amsfonts}
\usepackage{hyperref}

\setlength{\oddsidemargin}{.25in}
\setlength{\evensidemargin}{.25in}
\setlength{\textwidth}{6.25in}
\setlength{\topmargin}{-0.4in}
\setlength{\textheight}{8.5in}

\newcommand{\heading}[5]{
   \renewcommand{\thepage}{#1-\arabic{page}}
   \noindent
   \begin{center}
   \framebox[\textwidth]{
     \begin{minipage}{0.9\textwidth} \onehalfspacing
       {\bf CS 290G -- Introduction to Modern Cryptography} \hfill #2

       {\centering \Large #5
       
       }\medskip

       {\it #3 \hfill #4}
     \end{minipage}
   }
   \end{center}
}

\newcommand{\handout}[3]{\heading{#1}{#2}{Instructor:
Stefano Tessaro}{Student: Shiyu Ji, Yi Yang}{#3}}

\setlength{\parindent}{0in}

\newcommand{\eqdef}{\stackrel{def}{=}}
\newcommand{\N}{\mathbb{N}}
\newcommand{\R}{\mathbb{R}}
\newcommand{\Z}{\mathbb{Z}}
\newcommand{\F}{\mathbb{F}}
\newcommand{\bits}{\{0,1\}}
\newcommand{\inr}{\in_{\mbox{\tiny R}}}
%\newcommand{\getsr}{\gets_{\mbox{\tiny R}}}
\newcommand{\getsr}{\stackrel{\$}{\gets}}
\newcommand{\st}{\mbox{ s.t. }}
\newcommand{\etal}{{\it et al }}
\newcommand{\into}{\rightarrow}

\newcommand{\Ex}{\mathbb{E}}
\newcommand{\e}{\epsilon}
\newcommand{\ee}{\varepsilon}
\newcommand{\ceil}[1]{{\lceil{#1}\rceil}}
\newcommand{\floor}[1]{{\lfloor{#1}\rfloor}}
\newcommand{\angles}[1]{\langle #1 \rangle}
\newcommand{\Com}{{\sf Com}}
\newcommand{\desc}{{\sf desc}}

\newcommand{\rightstep}[1]{%
$\underrightarrow{\quad #1 \quad}$ }

\newcommand{\leftstep}[1]{%
$\underleftarrow{\quad #1 \quad}$ }
\newcommand{\Adv}{\textsf{Adv}}
\newcommand{\tab}{\hspace{0.3in}}

%%%%%%%%%%%%%%%%%%%%%%%%%%%%
% Theorems & Definitions


\newtheorem{theorem}{Theorem}[section]

\newtheorem{claim}[theorem]{Claim}
\newtheorem{subclaim}{Claim}[theorem]
\newtheorem{proposition}[theorem]{Proposition}
\newtheorem{lemma}[theorem]{Lemma}
\newtheorem{corollary}[theorem]{Corollary}
\newtheorem{conjecture}[theorem]{Conjecture}
\newtheorem{observation}[theorem]{Observation}

\theoremstyle{definition}
\newtheorem{definition}[theorem]{Definition}
\newtheorem{construction}[theorem]{Construction}
\newtheorem{example}[theorem]{Example}
\newtheorem{counterexample}[theorem]{Counterexample}
\newtheorem{algorithm1}[theorem]{Algorithm}
\newtheorem{protocol}[theorem]{Protocol}
\newtheorem{remark}[theorem]{Remark}
\newtheorem{assumption}[theorem]{Assumption}
\newtheorem{fact}[theorem]{Fact}

%\bibliographystyle{plain}

\begin{document}
\handout{2}{Due: Feb 12, 2016}{Homework 2}
\section{Task 1 - Pseudorandom Functions}
Let $F : \bits^\lambda \times \bits^n \to \bits^n$ be a PRF. As in class, we use the notation $F_s(x) = F(s, x)$.

{\bf a) [Points: 4]} Consider $F' : \bits^\lambda \times \bits^{2n} \to \bits^n$ such that
$$F'_s(x_1 || x_2) = F_s(x_1) \oplus F_s(x_2)$$
for all $x_1, x_2 \in \bits^n$ and $s\in\bits^\lambda$. Is $F'$ a PRF?

{\bf Claim}: $F'$ is \emph{not} a PRF.
\begin{proof}
It suffices to give a distinguisher $D$ s.t. $\Adv_{F'}^{prf}(D)$ is non-negligible.
\begin{quote}
Distinguisher $D^O$: // $O$ can be either $F'_s$ or $RF_{2n,n}$.
\begin{enumerate}
\item $c \gets O(0^n0^n)$.
\item {\bf if} $c=0^n$ {\bf then return} 1.
\item {\bf else return} 0.
\end{enumerate}
\end{quote}
Note that the oracle $F'_s$ always return $0^n$ on the query $(0^n || 0^n)$, whereas $RF_{2n,n}$ gives uniformly random answer. 
We can finish the proof by computing the advantage:
$$
\Adv_{F'}^{prf}(D) 
= \bigg| \Pr[s \getsr \bits^\lambda : D^{F'_s}=1] - \Pr[D^{RF_{2n,n}}=1] \bigg| \\
= 1 - 2^{-n},
$$
which is overwhelming.
\end{proof}

{\bf b) [Points: 6]} Consider $F'' : \bits^{2\lambda} \times \bits^{2n} \to \bits^n$ such that
$$F''_{s_1||s_2}(x_1||x_2) = F_{s_1}(x_1) \oplus F_{s_2}(x_2)$$
for all $x_1, x_2 \in \bits^n$ and $s_1,s_2 \in \bits^\lambda$. Is $F''$ a PRF?

{\bf Claim}: $F''$ is \emph{not} a PRF.
\begin{proof}
It suffices to give a distinguisher $D$ s.t. $\Adv_{F''}^{prf}(D)$ is non-negligible.
\begin{quote}
Distinguisher $D^O$: // $O$ can be either $F''_{s_1||s_2}$ or $RF_{2n,n}$.
\begin{enumerate}
\item $c_1 \gets O(0^n0^n)$.
\item $c_2 \gets O(0^n1^n)$.
\item $c_3 \gets O(1^n0^n)$.
\item $c_4 \gets O(1^n1^n)$.
\item {\bf if} $c_1 \oplus c_2 = c_3 \oplus c_4$ {\bf then output} 1.
\item {\bf else output} 0.
\end{enumerate}
\end{quote}
Note that if $O$ is $F''_{s_1||s_2}$, we have
$$c_1 \oplus c_2 = F_{s_2}(0^n) \oplus F_{s_2}(1^n) = c_3 \oplus c_4.$$
If $O$ is $RF_{2n,n}$, then each $c_i$ is independently uniformly random.
We estimate the advantage
$$
\Adv_{F''}^{prf}(D) 
= \bigg| \Pr[s \getsr \bits^\lambda : D^{F''_s}=1] - \Pr[D^{RF_{2n,n}}=1] \bigg| \\
= 1 - 2^{-n},
$$
which is overwhelming.
\end{proof}

{\bf c) [Points: 6]} Consider the keyed function $F''' : \bits^\lambda \times \bits^n \to \bits^n$ such that
$$F'''_s(x) = F_s(x) \oplus x$$
for all $s \in \bits^\lambda$ and $s_1, s_2 \in \bits^\lambda$. Is $F'''$ a PRF?

{\bf Claim}: $F'''$ is a PRF.
\begin{proof}
Assume by contradiction $F'''$ is not a PRF. Then there exists a distinguisher $D$ s.t. $\Adv_{F'''}^{prf}(D)$ is non-negligible. It suffices to build a distinguisher $D'$ s.t. $\Adv_{F}^{prf}(D')$ is also non-negligible. We give such a construction as following:
\begin{quote}
Distinguisher $D'^O$: // $O$ can be either $F_s$ or $RF_{n,n}$.
\begin{enumerate}
\item $b \getsr D^{\widetilde{O}}$.
\item {\bf return} $b$.
\end{enumerate}
\end{quote}
Here the oracle $\widetilde{O}$ is defined as the following procedure:
\begin{quote}
Procedure $\widetilde{O} (x)$: // query $x\in\bits^n$.
\begin{enumerate}
\item $r \gets O(x)$.
\item {\bf return} $r \oplus x$.
\end{enumerate}
\end{quote}
Note that if the oracle $O$ is $F_s$, then $\widetilde{O}$ simulates $F'''_s$. If $O$ is $RF_{n,n}$, the output of $\widetilde{O}$ is $RF_{n,n}(x) \oplus x$ on the query $x$. Since $RF_{n,n}$ always samples at uniformly random upon each new query, the distribution of $RF_{n,n}(x) \oplus x$ is also independently uniformly distributed for each distinct $x$, like OTP. Hence $\widetilde{O}$ simulates $RF_{n,n}$.
Formally, define a hybrid oracle $H(x) \eqdef RF_{n,n}(x) \oplus x$, and hence the output distribution of $H$ is identical to that of $RF_{n,n}$. Thus the advantage of $D'$ is
$$\begin{aligned}
&\Adv_{F}^{prf}(D') = \bigg| \Pr[s \getsr \bits^\lambda : D'^{F_s}=1] - \Pr[D'^{RF_{n,n}}=1] \bigg| \\
=& \bigg| \Pr[s \getsr \bits^\lambda : D^{F'''_s}=1] - \Pr[D^{H}=1] \bigg| \\
=& \bigg| \Pr[s \getsr \bits^\lambda : D^{F'''_s}=1] - \Pr[D^{RF_{n,n}}=1] \bigg| = \Adv_{F'''}^{prf}(D),
\end{aligned}$$
which is non-negligible as desired.
\end{proof}

\section{Task 2 - Feistel Construction}
Let $F : \bits^\lambda \times \bits^n \to \bits^n$ be a PRF. Recall that the $r$-round \emph{Feistel construction} uses $F$ to implement a block cipher $\Psi^r : \bits^{r\lambda} \times \bits^{2n} \to \bits^{2n}$. In particular, for key/seed $s = s_1 || s_2 || \cdots || s_r$ (where $s_1, \cdots, s_r \in \bits^\lambda$) and input $x = x_0 || x_1$ (where $x_0, x_1 \in \bits^n$), the Feistel construction outputs $\Psi_s^r(x) = x_r || x_{r+1}$, where for all $i=1, \cdots, r$, we have
$$x_{i+1} \gets F_{s_i}(x_i) \oplus x_{i-1}.$$

{\bf a) [Points: 4]} Show that $\Psi^r$ is indeed a block cipher, i.e., show that given $s = s_1 || s_2 || \cdots || s_r$, the map $\Psi_s^r(\cdot)$ is a permutation, and that its inverse is efficiently computable.

\begin{proof}
Since $\Psi_s^r$ preserves the length, to show it is a permutation, it suffices to show that it is one-to-one, i.e., $\Psi_s^r(x) = \Psi_s^r(y)$ implies $x = y$. If $\Psi_s^r(x) = \Psi_s^r(y)$, then $x_r = y_r$ and $x_{r+1} = y_{r+1}$. If $r=1$, then $\Psi_s^1$ is defined as $(x_0, x_1) \mapsto (x_1, F_{s_1}(x_1) \oplus x_{0})$, which is clearly injective since $(y_1,y_2) \mapsto (F_{s_1}(y_1)\oplus y_2, y_1)$ is its inverse. If $r \geq 2$, then from $x_{r+1} = y_{r+1}$ we have
$$F_{r}(x_{r}) \oplus x_{r-1} = F_{r}(y_{r}) \oplus y_{r-1}.$$ 
Then since $x_r = y_r$, $F_r(x_r) = F_r(y_r)$ and thus $x_{r-1} = y_{r-1}$. If $r-1=0$, then we have shown that $(x_0,x_1) = (y_0, y_1)$. Otherwise, use the same reasoning we can show that $x_{r} = y_{r}$ and $x_{r-1} = y_{r-1}$ imply $x_{r-2} = y_{r-2}$. After at most $r$ rounds we reach $x_0 = y_0$.

An alternative way to show the bijection is by giving its inverse function. The inverse of $\Psi_s^r(\cdot)$ can be computed efficiently as following: given input $(y_r, y_{r+1})$, we output $(y_0, y_1)$ s.t. for $i$ from $r$ to $1$, we compute
$$y_{i-1} \gets F_{s_i}(y_i) \oplus y_{i+1}.$$
It is easy to see the value given by the PRF $F$ is canceled out by XORing.
\end{proof}

{\bf b) [Points: 6]} Prove that $\Psi^2$ is \emph{not} a pseudorandom permutation (PRP).
\begin{proof}
Given $s = (s_1, s_2)$, $\Psi^2$ is given by 
$$\Psi_s^2 : (x_0,x_1) \mapsto (F_{s_1}(x_1) \oplus x_{0}, F_{s_2}(F_{s_1}(x_1) \oplus x_{0}) \oplus x_1).$$
To show that $\Psi^2$ is not PRP, it suffices to build a distinguisher $D$ s.t. $\Adv_{\Psi^2}^{prp}(D)$ is non-negligible. We give such a construction as following:
\begin{quote}
Distinguisher $D^O$: // $O$ can be either $\Psi_s^2$ or $RP_{2n}$.
\begin{enumerate}
\item $(c_1,c_2) \gets O(1^n0^n)$.
\item $(c_3,c_4) \gets O(0^n0^n)$.
\item {\bf if} $c_1 \oplus c_3 = 1^n$ {\bf then output} 1.
\item {\bf else output} 0.
\end{enumerate}
\end{quote}
Note that if $O$ is $\Psi_s^2$, then $c_1 = F_{s_1}(0^n) \oplus 1^n$ and $c_3 = F_{s_1}(0^n) \oplus 0^n$ and thus $c_1 \oplus c_3 = 1^n$. If $O$ is $RP_{2n}$, then $c_1$ and $c_3$ are independently uniformly sampled. Hence the advantage of $D$ is
$$\Adv_{\Psi^2}^{prp}(D) = \bigg| \Pr[s \getsr \bits^\lambda : D^{\Psi_s^2} = 1] - \Pr[D^{RP_{2n}}=1] \bigg| = 1 - 2^{-n},$$
which is overwhelming.
\end{proof}

\section{Task 3 - Collision Probabilities}
Let $X_1, \cdots, X_q$ be uniformly and independently sampled from a set $S$ with $N$ elements. We are interested in the probability $p_c(q, N)$ that there exist indices $i \not = j$ such that $X_i = X_j$. In class, we have proved that $p_c(q, N) \leq \frac{q^2}{2N}$.

{\bf a) [Points: 10]} Prove that
$$p_c(q, N) \geq 1-e^{-\frac{q(q-1)}{2N}}.$$
\begin{proof}
Define the event $E$: there exist indices $i \not = j$ such that $X_i = X_j$. 
Consider its complementary event $\overline{E}$: for any distinct indices $i \not= j$, we have $X_i \not= X_j$. In other words, there is no collision among the $X$ values. We rewrite its probability as
$$\Pr[\overline{E}] = \Pr[\bigwedge_{1\leq i < j \leq q} : X_i \not = X_j].$$
Note that the events $X_i \not = X_j$ and $X_{i'} \not = X_{j'}$ are independent if $\{i,j\} \not= \{i',j'\}$. To see this, we can compute
$$\Pr[X_i \not = X_j \wedge X_{i'} \not = X_{j'}] = \Pr[X_i \not = X_j]\Pr[X_{i'} \not = X_{j'} | X_i \not = X_j].$$
Note that each $X_i$ is sampled uniformly and independently. If $i, j, i', j'$ are all distinct, it is clear that
$$\Pr[X_{i'} \not = X_{j'} | X_i \not = X_j] = 1-\frac{1}{N} = \Pr[X_{i'} \not = X_{j'} ].$$
Otherwise, w.l.o.g. we assume $i=i'$ and $j \not= j'$, then we have
$$\Pr[X_{i'} \not = X_{j'} | X_i \not = X_j] = \Pr[X_{i'} \not = X_{j'} | X_{i'} \not = X_j] = 1-\frac{1}{N} = \Pr[X_{i'} \not = X_{j'} ].$$
For either case we have
$$\Pr[X_i \not = X_j \wedge X_{i'} \not = X_{j'}] = \Pr[X_i \not = X_j]\Pr[X_{i'} \not = X_{j'}].$$
Hence for $\overline{E}$, we have
$$\Pr[\overline{E}] = \prod_{1\leq i < j \leq q}\Pr[X_i \not = X_j] = \left( 1 - \frac{1}{N} \right)^{\binom{q}{2}}.$$
Here we need to use a bound:
\begin{lemma}
\label{lem:1}
$1-\frac{1}{N} \leq e^{-\frac{1}{N}}$ for all $N\geq 1$.
\end{lemma}
\begin{proof}
The function $f(x) \eqdef e^x-1-x$ achieves its minimum value 0 at $x=0$, since its derivative $f'(x) = e^x-1$ has $f'(0)=0$ and $f(x)$ is convex. Let $x=-\frac{1}{N}$ and the lemma follows.
\end{proof}
By using Lemma \ref{lem:1}, we have
$$\Pr[\overline{E}] \leq e^{-\frac{1}{N}\binom{q}{2}} = e^{-\frac{q(q-1)}{2N}},$$
and $\Pr[E] = 1-\Pr[\overline{E}]$ finishes the proof.
\end{proof}

{\bf b) [Points: 2]} Show that if $q \leq \sqrt{2N}$, then 
$$p_c(q,N) \geq c \cdot \frac{q(q-1)}{N}$$
for some constant $c$.
\begin{proof}
We first need a lemma.
\begin{lemma}
$\left( 1-\frac{1}{e} \right)x \leq 1-e^{-x}$ for all $x \in [0,1]$.
\end{lemma}
\begin{proof}
Consider the function $f(x) \eqdef e^{-x} + \left( 1-\frac{1}{e} \right)x-1$. Note that $f(0) = f(1) = 0$ and $f$ is convex. Hence $f(x) \leq 0$ for all $x \in [0,1]$ and the lemma follows.
\end{proof}
Based on the above lemma and a), we have
$$p_c(q, N) \geq 1-e^{-\frac{q(q-1)}{2N}} \geq \left( 1-\frac{1}{e} \right)\frac{q(q-1)}{2N}.$$
Set $c = \frac{1}{2} - \frac{1}{2e}$ and b) follows.
\end{proof}


\section{Task 4 - Key Recovery}
{\bf [10 Points]} A basic property one would expect from a good symmetric encryption scheme is that \emph{key
recovery} is hard, i.e., it is hard to recover the secret key, even when the adversary is able to
encrypt messages of her choice.

Formalize key recovery security with an appropriate advantage metric, and sketch a proof
that ind-cpa security implies your security notion.

\begin{definition}
A symmetric encryption $SE = (KG, Enc, Dec)$ with message space $M = \{M_\lambda\}_{\lambda \in \N}$ is {\bf key recovery secure} if for all PPT adversary $A$, for all $\lambda$, the advantage defined by
$$\Adv_{SE}^{key-rec}(A) \eqdef \Pr[k \getsr KG(1^\lambda), k' \getsr A^{Enc(k,\cdot)} : k' = k]$$
is negligible. 
\end{definition}

\begin{lemma}
Ind-cpa security implies key recovery security.
\end{lemma}
We first give a proof with strong correctness notion, and then fix that for the relaxed version.
\begin{proof}
Given a symmetric encryption scheme $SE$, assume by contradiction $SE$ is not key recovery secure, then there exists an adversary $A$ s.t. $\Adv_{SE}^{key-rec}(A)$ is non-negligible. Choose two distinct messages $m, m' \in M_\lambda$ s.t. $m \not= m'$. It suffices to build a PPT distinguisher $D$ s.t. $\Adv_{SE}^{ind-cpa}(D)$ is also non-negligible. We give the construction as following:
\begin{quote}
Distinguisher $D^O$: // $O$ can be either $LR_0^k$ or $LR_1^k$.
\begin{enumerate}
\item $k' \getsr A^{Enc(k,\cdot)}$.
\item $c_1 \gets O(m, m)$.
\item $c_2 \gets O(m, m')$.
\item {\bf if} $Dec(k', c_1) = m$ {\bf then } 
\item \tab {\bf if} $Dec(k', c_2) = m$ {\bf then return} 1.
\item \tab {\bf else return} 0.
\item {\bf else return} 0.
\end{enumerate}
\end{quote}
Now we estimate the advantage
$$\begin{aligned}
&\Adv_{SE}^{ind-cpa}(D) \\
=& \bigg| \Pr[k \getsr KG(1^\lambda) : D^{LR_0^k}=1] - \Pr[k \getsr KG(1^\lambda) : D^{LR_1^k}=1] \bigg| \\
=& \bigg| \Pr[k \getsr KG(1^\lambda), k' \getsr A^{Enc(k,\cdot)}: Dec(k', Enc(k, m))=m] \\
&-\Pr[k \getsr KG(1^\lambda), k' \getsr A^{Enc(k,\cdot)}: Dec(k', Enc(k, m))=m \wedge Dec(k', Enc(k, m'))=m] \bigg|.
\end{aligned}$$
Hence by the strong correctness assumption of symmetric encryption 
$$\Pr[Dec(k, Enc(k, m))=m]=1,$$ 
we have
$$\begin{aligned}
&\Adv_{SE}^{ind-cpa}(D) \\
=& \Pr[k \getsr KG(1^\lambda), k' \getsr A^{Enc(k,\cdot)}: Dec(k', Enc(k, m))=m \wedge Dec(k', Enc(k, m'))\not=m] \\
\geq & \Pr[k \getsr KG(1^\lambda), k' \getsr A^{Enc(k,\cdot)}: k' = k \wedge Dec(k', Enc(k, m'))\not=m].
\end{aligned}$$
By conditional probability,
$$\begin{aligned}
&\Pr[k \getsr KG(1^\lambda), k' \getsr A^{Enc(k,\cdot)}: k' = k \wedge Dec(k', Enc(k, m'))\not=m] \\
=&\Pr[k \getsr KG(1^\lambda), k' \getsr A^{Enc(k,\cdot)}: k' = k] \cdot \Pr[Dec(k', Enc(k, m'))\not=m | k' = k].
\end{aligned}$$
By correctness of SE, we know $Dec(k', Enc(k, m')) = m' \not= m$ for sure if $k'=k$. Thus
$$\begin{aligned}
&\Adv_{SE}^{ind-cpa}(D) 
\geq  \Pr[k \getsr KG(1^\lambda), k' \getsr A^{Enc(k,\cdot)}: k' = k] 
= \Adv_{SE}^{key-rec}(A),
\end{aligned}$$
which is non-negligible.
\end{proof}
Note that the proof above assumes the notion of SE correctness as $k=k'$ implies $Dec(k', Enc(k, m)) = m$ for any $m$. If negligible errors are allowed, we can fix the proof as following: define $K$ as the set of all the keys $k$ s.t. $Dec(k, Enc(k, m)) = m$ for any $m$. 
Then
$$\begin{aligned}
&\Adv_{SE}^{ind-cpa}(D) \\
=& \bigg| \Pr[k \getsr KG(1^\lambda) : k \in K]\cdot(\Pr[k\getsr K : D^{LR_0^k}=1] - \Pr[k\getsr K : D^{LR_1^k}=1]) \\
& + \Pr[k \getsr KG(1^\lambda) : k \not\in K]\cdot(\Pr[k\not\in K : D^{LR_0^k}=1] - \Pr[k\not\in K : D^{LR_1^k}=1]) \bigg|.
\end{aligned}$$
Since $\Pr[k \getsr KG(1^\lambda) : k \not\in K]$ is negligible by the relaxed correctness notion, it suffices to show
$$P \eqdef \bigg| \Pr[k\getsr K : D^{LR_0^k}=1] - \Pr[k\getsr K : D^{LR_1^k}=1] \bigg|$$
is non-negligible.
A similar reasoning shows that
$$P
\geq \Pr[k \getsr K, k' \getsr A^{Enc(k,\cdot)}: k' = k] 
\geq \Adv_{SE}^{key-rec}(A).$$

\section{Task 5 - Concrete Security for Symmetric Encryption}
We want to have a closer look at the quantitative impact of concrete attacks on the security
of symmetric encryption schemes. In particular, consider the randomized counter-mode
encryption scheme CTR with message space $M = \bits^\lambda$ as presented in class, using a
PRF $F : \bits^\lambda \times \bits^\lambda \to \bits^\lambda$, it chooses a random $r \getsr \bits^\lambda$
and produces the ciphertext $c = (r, F(k, r) \oplus m)$.

{\bf a) [Points: 5]} For every $q$, give a concrete distinguisher $D$, making $q$ queries to its oracle, achieving advantage
$\Adv_{CTR}^{ind-cpa}(D) = \Omega(q^2/2^\lambda)$.
\begin{proof}
The idea of such a $D$ is from birthday attack:
\begin{quote}
Distinguisher $D^O$: // $O$ can be either $LR_0^k$ or $LR_1^k$.
\begin{enumerate}
\item {\bf for} $i=1$ to $q$ {\bf do}
\item \tab uniformly and independently choose two messages $m_i, n_i$.% s.t. $m_i \not= m_j$ and $n_i \not= n_j$ for any $j<i$.
\item \tab $(r_i, c_i) \gets O(m_i, n_i)$.
\item {\bf if} there exist $i\not= j$ s.t. $r_i = r_j$ {\bf then}
\item \tab {\bf if} $m_i \oplus m_j = c_i \oplus c_j$ {\bf then return} 1.
\item \tab {\bf else} {\bf return} 0.
\item {\bf else} $b\getsr\bits$ and {\bf return} b.
\end{enumerate}
\end{quote}
It is easy to compute the advantage 
$$\begin{aligned}
& \Adv_{CTR}^{ind-cpa}(D) \\
=& \bigg| \Pr[k \getsr KG(1^\lambda) : D^{LK_0^k}=1] - \Pr[k \getsr KG(1^\lambda) : D^{LK_1^k}=1] \bigg| \\
=& \bigg| \Pr[k \getsr KG(1^\lambda) : D^{LK_0^k}=1 \wedge \textrm{$r$ collides}] - \Pr[k \getsr KG(1^\lambda) : D^{LK_1^k}=1 \wedge \textrm{$r$ collides}] \\
&+\Pr[k \getsr KG(1^\lambda) : D^{LK_0^k}=1 \wedge \textrm{no collision}] - \Pr[k \getsr KG(1^\lambda) : D^{LK_1^k}=1 \wedge \textrm{no collision}] \bigg| \\
\end{aligned}$$
Since if no collision, $D$ with always output 1 with probability 1/2,
$$\begin{aligned}
& \Adv_{CTR}^{ind-cpa}(D) \\
=& \bigg| \Pr[k \getsr KG(1^\lambda) : D^{LK_0^k}=1 \wedge \textrm{$r$ collides}] - \Pr[k \getsr KG(1^\lambda) : D^{LK_1^k}=1 \wedge \textrm{$r$ collides}] \bigg|. \\
\end{aligned}$$
For the first probability term,
$$\begin{aligned}
& \Pr[k \getsr KG(1^\lambda) : D^{LK_0^k}=1 \wedge \textrm{$r$ collides}] \\
=& \Pr[k \getsr KG(1^\lambda): \textrm{$r$ collides}]\Pr[\textrm{$r_i, r_j$ collide} : m_i \oplus m_j = c_i \oplus c_j] \\
=& \Pr[k \getsr KG(1^\lambda): \textrm{$r$ collides}].
\end{aligned}$$
For the second probability term,
$$\begin{aligned}
& \Pr[k \getsr KG(1^\lambda) : D^{LK_1^k}=1 \wedge \textrm{$r$ collides}] \\
=& \Pr[k \getsr KG(1^\lambda): \textrm{$r$ collides}]\Pr[\textrm{$r_i, r_j$ collide} : m_i \oplus m_j = c_i \oplus c_j]. \\
\end{aligned}$$
However, this time 
$$m_i \oplus m_j = c_i \oplus c_j = F(k, r_i) \oplus n_i \oplus F(k, r_j) \oplus n_j = n_i \oplus n_j,$$
which happens with probability $2^{-\lambda}$, since the samplings of $m$ and $n$ are uniformly independent. \footnote{See Task 5 c) for more general discussion.}
Thus
$$\begin{aligned}
& \Pr[k \getsr KG(1^\lambda) : D^{LK_1^k}=1 \wedge \textrm{$r$ collides}] \\
=& \Pr[k \getsr KG(1^\lambda): \textrm{$r$ collides}]\cdot 2^{-\lambda}. 
\end{aligned}$$
Hence 
$$\begin{aligned}
&\Adv_{CTR}^{ind-cpa}(D) = (1-2^{-\lambda}) \Pr[k \getsr KG(1^\lambda): \textrm{$r$ collides}] \\
\geq & (1-2^{-\lambda})(1-e^{-\frac{q(q-1)}{2\cdot 2^\lambda}}) \geq (1-2^{-\lambda})(1-\frac{1}{e})\frac{q(q-1)}{2\cdot 2^\lambda}
= \Omega(q^2/2^\lambda).
\end{aligned}$$
\end{proof}

{\bf b) [Points: 2]} Assume we instantiate $F$ in counter-mode with a short $\lambda$ (e.g., using DES, where $\lambda = 64$). Would this be a good idea?

\emph{Solution}: It is not a good idea since by doing this only $q = 2^{32}$ queries will break the scheme with significant advantage no less than
$$(1-2^{-\lambda})(1-\frac{1}{e})\frac{q(q-1)}{2\cdot 2^\lambda} \bigg|_{\lambda = 64, q = 2^{32}} \approx 0.316.$$
Note that $2^{32}$ is not big enough for modern computers.

{\bf c) [Points: 5]} Consider the following (less efficient) variant of counter-mode encryption.
To encrypt a message $m \in \bits^\lambda$ with secret key $k \in \bits^\lambda$, it chooses random strings $r_1, \cdots, r_{10} \getsr \bits^\lambda$ and produces the ciphertext $c = (r_1, \cdots, r_{10}, F(k, r_1) \oplus \cdots \oplus F(k, r_{10}) \oplus m$. (Note that the ciphertext has now length $11\lambda$ bits.)

Try to give the best possible $q$-query distinguisher you can find against this construction, and analyze its ind-cpa advantage.

\begin{claim}
The less-efficient variant above cannot improve the ind-cpa security of CTR.
\end{claim}
\begin{proof}
We can construct a distinguisher that can achieve almost identical ind-cpa advantage, i.e., $\Omega(q^2/2^\lambda)$.
The idea of such a distinguisher $D$ is similar to a), but in addition we compute a value $R_i$ from $m_i$ and $c_i$:
\begin{quote}
Distinguisher $D^O$: // $O$ can be either $LR_0^k$ or $LR_1^k$.
\begin{enumerate}
\item {\bf for} $i=1$ to $q$ {\bf do}
\item \tab uniformly and independently choose two messages $m_i, n_i$.% s.t. $m_i \not= m_j$ and $n_i \not= n_j$ for any $j<i$.
\item \tab $(r_{i,1}, \cdots, r_{i,10}, c_i) \gets O(m_i, n_i)$.
\item {\bf for} $i=1$ to $q$ {\bf do}
\item \tab $R_i \gets c_i \oplus m_i$.
\item {\bf if} there exist $i\not= j$ s.t. $R_i = R_j$ {\bf then}
\item \tab {\bf if} $m_i \oplus m_j = c_i \oplus c_j$ {\bf then return} 1.
\item \tab {\bf else} {\bf return} 0.
\item {\bf else} $b\getsr\bits$ and {\bf return} b.
\end{enumerate}
\end{quote}
Note that if $O$ is $LR_0^k$, then each $R_i = F(k, r_{i,1}) \oplus \cdots \oplus F(k, r_{i,10})$. Hence if there is a collision $R_i = R_j$ for $i \not= j$, then we have $m_i \oplus m_j = c_i \oplus c_j$ if $O$ is $LR_0^k$.

We can compute the advantage 
$$\begin{aligned}
& \Adv_{CTR}^{ind-cpa}(D) \\
=& \bigg| \Pr[k \getsr KG(1^\lambda) : D^{LK_0^k}=1] - \Pr[k \getsr KG(1^\lambda) : D^{LK_1^k}=1] \bigg| \\
=& \bigg| \Pr[k \getsr KG(1^\lambda) : D^{LK_0^k}=1 \wedge \textrm{$R$ collides}] - \Pr[k \getsr KG(1^\lambda) : D^{LK_1^k}=1 \wedge \textrm{$R$ collides}] \\
&+\Pr[k \getsr KG(1^\lambda) : D^{LK_0^k}=1 \wedge \textrm{no collision}] - \Pr[k \getsr KG(1^\lambda) : D^{LK_1^k}=1 \wedge \textrm{no collision}] \bigg| \\
\end{aligned}$$
Since if no collision, $D$ with always output 1 with probability 1/2,
$$\begin{aligned}
& \Adv_{CTR}^{ind-cpa}(D) \\
=& \bigg| \Pr[k \getsr KG(1^\lambda) : D^{LK_0^k}=1 \wedge \textrm{$R$ collides}] - \Pr[k \getsr KG(1^\lambda) : D^{LK_1^k}=1 \wedge \textrm{$R$ collides}] \bigg|. \\
\end{aligned}$$
For the first probability term,
$$\begin{aligned}
& \Pr[k \getsr KG(1^\lambda) : D^{LK_0^k}=1 \wedge \textrm{$R$ collides}] \\
=& \Pr[k \getsr KG(1^\lambda): \textrm{$R$ collides}]\Pr[\textrm{$R_i, R_j$ collide} : m_i \oplus m_j = c_i \oplus c_j] \\
=& \Pr[k \getsr KG(1^\lambda): \textrm{$R$ collides}].
\end{aligned}$$
For the second probability term,
$$\begin{aligned}
& \Pr[k \getsr KG(1^\lambda) : D^{LK_1^k}=1 \wedge \textrm{$R$ collides}] \\
=& \Pr[k \getsr KG(1^\lambda): \textrm{$R$ collides}]\Pr[\textrm{$R_i, R_j$ collide} : m_i \oplus m_j = c_i \oplus c_j]. \\
\end{aligned}$$
However, since this time $O$ is $LK_1^k$, for each $i$, $R_i = F(k, r_{i,1}) \oplus \cdots \oplus F(k, r_{i,10}) \oplus n_i \oplus m_i$, and thus 
$$\begin{aligned}
&m_i \oplus m_j = c_i \oplus c_j = R_i \oplus m_i \oplus R_j \oplus m_j \\
=& F(k, r_{i,1}) \oplus \cdots \oplus F(k, r_{i,10}) \oplus n_i \oplus F(k, r_{j,1}) \oplus \cdots \oplus F(k, r_{j,10}) \oplus n_j,
\end{aligned}$$
which happens with probability $2^{-\lambda}$, since the samplings of $m$, $n$, $r_{i,\cdot}$ and $r_{j,\cdot}$ are uniformly independent. To see this, one may fix all the random variables except $m_i$ firstly, and see that the probability that
$$\begin{aligned}
& m_i = m_j \oplus c_i \oplus c_j \\
=& m_j \oplus F(k, r_{i,1}) \oplus \cdots \oplus F(k, r_{i,10}) \oplus n_i \oplus F(k, r_{j,1}) \oplus \cdots \oplus F(k, r_{j,10}) \oplus n_j
\end{aligned}$$
is $2^{-\lambda}$. This conditional probability holds for each combination of the random variables except $m_i$. Thus the total probability is also $2^{-\lambda}$. 
Hence
$$\begin{aligned}
& \Pr[k \getsr KG(1^\lambda) : D^{LK_1^k}=1 \wedge \textrm{$R$ collides}] \\
=& \Pr[k \getsr KG(1^\lambda): \textrm{$R$ collides}]\cdot 2^{-\lambda}. 
\end{aligned}$$
Since each $m_i$, $r_{i,\cdot}$ is independently and uniformly distributed, each $R_i = m_i \oplus c_i$ is also independently and uniformly distributed. Hence we can still apply birthday attack:
$$\begin{aligned}
& \Adv_{CTR}^{ind-cpa}(D) = (1-2^{-\lambda}) \Pr[k \getsr KG(1^\lambda): \textrm{$R$ collides}] \\
\geq & (1-2^{-\lambda})(1-e^{-\frac{q(q-1)}{2\cdot 2^\lambda}}) \geq (1-2^{-\lambda})(1-\frac{1}{e})\frac{q(q-1)}{2\cdot 2^\lambda}
= \Omega(q^2/2^\lambda).
\end{aligned}$$
Compared to a), the advantage of our new distinguisher does not decrease.
\end{proof}

%\begin{proof}
%Denote by $Q$ the set of the $q$ queries. The answer to each query $i \in Q$ can be represented as a vector $(r_{i,1}, \cdots, r_{i,10}, c_i)$. 
%Suppose there is a multigraph $G = (Q, E)$ allowing self loops. Each edge between two nodes $i_1, i_2$ (can be the same) is associated with two $r$ values $r_{i_1,j_1}$ and $r_{i_2,j_2}$ s.t. $r_{i_1,j_1} = r_{i_2,j_2}$ and $(i_1,j_1) \not= (i_2, j_2)$. Each $r$ value $r_{i,j}$ can only be associated with at most one edge. 
%The key observation is that if there is a subset $I \subseteq Q$ s.t. $I$ is a 10-regular subgraph, then 
%$$\bigoplus_{i \in I, 1\leq j \leq 10}F(k, r_{i,j}) = 0.$$
%Since the degree of each vertex in $Q$ is at most 10, such a subset $I$ must be a component of the graph $G$. 
%Hence by traversing the graph $G$ we can find such a 10-regular component if any. 
%Based on the idea above, we construct the distinguisher as following:
%
%\begin{quote}
%Distinguisher $D^O$: // $O$ can be either $LR_0^k$ or $LR_1^k$.
%\begin{enumerate}
%\item set up a $q\times 10$ table $T$, and a multigraph $G$ with $q$ vertices and no edges initially.
%\item set $T[i,j]=\bot$ for any $1\leq i\leq q$, $1\leq j \leq 10$.
%\item {\bf for} $i=1$ to $q$ {\bf do}
%\item \tab choose messages $m_i, n_i$ s.t. $m_i \not= m_j$ and $n_i \not= n_j$ for any $j<i$.
%\item \tab $(r_{i,1}, \cdots, r_{i,10}, c_i) \gets O(m_i, n_i)$.
%\item \tab {\bf if} there exists $r_{i,j} = r_{i',j'}$ s.t. $(i,j)\not=(i',j')$, $i\geq i'$, $T[i,j]=T[i',j']=\bot$ {\bf then}
%\item \tab \tab $T[i,j] \gets set, T[i',j'] \gets set$.
%\item \tab \tab add an edge between the vertices $i$ and $i'$ in $G$.
%\item traverse the graph to find if any 10-regular component $I$ exists.
%\item {\bf if} there exists such a component $I$ {\bf then}
%\item \tab {\bf if} $\bigoplus_{i \in I, 1\leq j \leq 10} m_i = \bigoplus_{i \in I, 1\leq j \leq 10} c_i$ {\bf then return} 1.
%\item \tab {\bf else} {\bf return} 0.
%\item {\bf else} $b\gets\bits$ and {\bf return} b.
%\end{enumerate}
%\end{quote}
%Note that $D$ is efficient, since both the number of queries and graph traversal are bounded by $q$.
%The challenging question is to calculate the advantage $\Adv_{CTR}^{ind-cpa}(D)$.
%By a similar reasoning in a), we can show that
%$$\Adv_{CTR}^{ind-cpa}(D) = (1-2^\lambda)\Pr[\textrm{$G$ has a 10-regular component.}].$$
%We give a lower bound on $P \eqdef \Pr[\textrm{$G$ has a 10-regular component.}]$.
%Clearly $P \geq \Pr[\textrm{$G$ has a 10-regular component of size 1.}]$.
%For any query $i \in Q$, the probability $q$ that $\{i\}$ is already a 10-regular component is
%$$q = \frac{\binom{10}{5}(2^\lambda)^5}{(2^\lambda)^{10}} = \frac{252}{2^{5\lambda}}.$$
%Thus $P = 1-(1-q)^{10} \geq 1-e^{-10q}=1-e^{-\frac{2520}{2^{5\lambda}}}.$
%We can also find a rough upper bound based on the idea that there must be at least 5 collisions. That is, we need to find the probability that there are at least 5 pairs of collisions among the $r_{i,j}$'s of size $10q$. Clearly by birthday attack, $P$ is bounded by
%$$P \leq \frac{(10q)^2}{2\cdot 2^\lambda} = \frac{50q^2}{2^\lambda}.$$
%\end{proof}
\end{document}
