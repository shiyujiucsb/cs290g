\documentclass[12pt]{article}
\usepackage{url,amsmath,setspace,amssymb,amsthm,amsfonts}


\setlength{\oddsidemargin}{.25in}
\setlength{\evensidemargin}{.25in}
\setlength{\textwidth}{6.25in}
\setlength{\topmargin}{-0.4in}
\setlength{\textheight}{8.5in}

\newcommand{\heading}[5]{
   \renewcommand{\thepage}{#1-\arabic{page}}
   \noindent
   \begin{center}
   \framebox[\textwidth]{
     \begin{minipage}{0.9\textwidth} \onehalfspacing
       {\bf CS 290G -- Introduction to Modern Cryptography} \hfill #2

       {\centering \Large #5
       
       }\medskip

       {\it #3 \hfill #4}
     \end{minipage}
   }
   \end{center}
}

\newcommand{\handout}[3]{\heading{#1}{#2}{Instructor:
Stefano Tessaro}{Scribe: Shiyu Ji}{Lecture #1: #3}}

\setlength{\parindent}{0in}

\newcommand{\eqdef}{\stackrel{def}{=}}
\newcommand{\N}{\mathbb{N}}
\newcommand{\R}{\mathbb{R}}
\newcommand{\Z}{\mathbb{Z}}
\newcommand{\F}{\mathbb{F}}
\newcommand{\bits}{\{0,1\}}
\newcommand{\inr}{\in_{\mbox{\tiny R}}}
%\newcommand{\getsr}{\gets_{\mbox{\tiny R}}}
\newcommand{\getsr}{\stackrel{\$}{\gets}}
\newcommand{\st}{\mbox{ s.t. }}
\newcommand{\etal}{{\it et al }}
\newcommand{\into}{\rightarrow}

\newcommand{\Ex}{\mathbb{E}}
\newcommand{\e}{\epsilon}
\newcommand{\ee}{\varepsilon}
\newcommand{\ceil}[1]{{\lceil{#1}\rceil}}
\newcommand{\floor}[1]{{\lfloor{#1}\rfloor}}
\newcommand{\angles}[1]{\langle #1 \rangle}
\newcommand{\Com}{{\sf Com}}
\newcommand{\desc}{{\sf desc}}

\newcommand{\rightstep}[1]{%
$\underrightarrow{\quad #1 \quad}$ }

\newcommand{\leftstep}[1]{%
$\underleftarrow{\quad #1 \quad}$ }

\newcommand{\Adv}{\mathsf{Adv}}

\newcommand{\tab}{\hspace{0.3in}}
%%%%%%%%%%%%%%%%%%%%%%%%%%%%
% Theorems & Definitions


\newtheorem{theorem}{Theorem}[section]

\newtheorem{claim}[theorem]{Claim}
\newtheorem{subclaim}{Claim}[theorem]
\newtheorem{proposition}[theorem]{Proposition}
\newtheorem{lemma}[theorem]{Lemma}
\newtheorem{corollary}[theorem]{Corollary}
\newtheorem{conjecture}[theorem]{Conjecture}

\theoremstyle{definition}
\newtheorem{definition}[theorem]{Definition}
\newtheorem{construction}[theorem]{Construction}
\newtheorem{example}[theorem]{Example}
\newtheorem{algorithm1}[theorem]{Algorithm}
\newtheorem{protocol}[theorem]{Protocol}
\newtheorem{remark}[theorem]{Remark}
\newtheorem{observation}[theorem]{Observation}
\newtheorem{assumption}[theorem]{Assumption}
\newtheorem{fact}[theorem]{Fact}

%\bibliographystyle{plain}
\usepackage{tikz}
\usetikzlibrary{calc,decorations.pathreplacing}

\begin{document}
\handout{6}{Jan 26, 2016}{Pseudorandom Permutations}
\section{Recap}
A function $F : \bits^\lambda \times \bits^n \to \bits^m$ is a PRF if
\begin{enumerate}
\item It is efficiently computable.
\item For any PPT distinguisher $D$, the advantage $\Adv_{F,\lambda}^{prf}(D)$ is negligible. $\Adv_{F,\lambda}^{prf}(D)$ is defined by
\[\Adv_{F,\lambda}^{prf}(D) \eqdef \bigg| \Pr[s \getsr \bits^\lambda : D^{F(s,\cdot)}=1] - \Pr[D^{RF_{n,m}}=1] \bigg|. \]
Here $D^{RF_{n,m}}$ denotes that $D$ can make queries to the black box $RF_{n,m}$ s.t.
\begin{itemize}
\item For every \emph{new} $x_i$, the box answers $y_i \getsr \bits^m$.
\item When $D$ asks two identical queries $x_i = x_j$, the box gives the same answer.
\end{itemize}
$D^{F(s,\cdot)}$ is similar except that the box answers $y_i \gets F(s,x_i)$ upon query $x_i$.
\end{enumerate}

\section{Block Cipher}
\begin{definition}
A function $E : \bits^\lambda \times \bits^n \to \bits^n$ is a \emph{block cipher} if
\begin{enumerate}
\item $E$ is efficiently computable.
\item For all $s \getsr \bits^\lambda$, the function $E_s : \bits^n \to \bits^n$ mapping $m \in \bits^n$ to $E(s,m)$ is a \emph{permutation}, i.e., one-to-one.
\item The inverse function $E^{-1}_s$ that maps $(s, y)$ to $E^{-1}_s(y)$ is also efficiently computable.
\end{enumerate}
\end{definition}

\begin{example}
There are several practical constructions of block cipher, e.g., Data Encryption Standard (DES), Advanced Encryption Standard (AES), etc. Both DES and AES are standardized block ciphers.

Most practical constructions of block cipher have fixed parameters, e.g., for AES, we have $\lambda = 128$ or $256$, $n=128$; for DES, we have $\lambda = 56$, $n=64$.
\end{example}

\section{Pseudorandom Permutation}
The next question is: what does it mean for a block cipher $E$ to be secure?

\emph{First try}: Is $E$ a PRF? Is ``random function'' the best abstraction for what a secure block cipher should achieve?

Unfortunately, when $n$ is relatively small, this goal cannot be achieved. For example, if $n=4$, then $E$ cannot be a PRF, since we have a distinguisher to tell it apart from random function. Concretely, the distinguisher $D$ may query all the 16 possible inputs and check whether all the outputs are distinct. If so, output 1. Otherwise, output 0. The advantage of this distinguisher $D$ is
$$
\Adv_{E}^{prf}(D) = \bigg|\Pr[s\getsr\bits^\lambda : D^{E(s,\cdot)}=1]-\Pr[D^{RF_{4,4}}=1] \bigg| = 1 - 16!/16^{16},
$$
which is very close to 1.

\emph{Second try}: Can we replace ``random function'' by ``random permutation''? Firstly we need to define random permutation.
\begin{definition}
Random permutation $RP_n$ is a system that uses a table $T$ to keep the states of each query. 
\begin{itemize}
\item Initially, set $T[x] \gets \bot$ for all $x \in \bits^n$.
\item For any query $x$, $RP_n(x)$ is defined as
\begin{quote}
Procedure $RP_n(x)$:
\begin{enumerate}
\item {\bf if} $T[x]=\bot$ {\bf then}
\item \tab $T[x] \getsr \{y \in \bits^n | \forall x' \in \bits^n, T[x'] \not= y\}.$
\item {\bf return} $T[x]$.
\end{enumerate}
\end{quote}
\end{itemize}
\end{definition}

A block cipher is a \emph{pseudorandom permutation} if we cannot tell it apart from a random permutation.

\begin{definition}
A block cipher $E$ is a \emph{pseudorandom permutation} (PRP) if for any PPT distinguisher $D$, the advantage $\Adv_{E,\lambda}^{prp}(D)$ is negligible. $\Adv_{E,\lambda}^{prp}(D)$ is defined as
$$\Adv_{E,\lambda}^{prp}(D) \eqdef \bigg| \Pr[s \getsr \bits^\lambda : D^{E(s,\cdot)} = 1] - \Pr[D^{RP_n}=1] \bigg|.$$
\end{definition}

We need to compare the notions of PRP and PRF, i.e., to give an upper bound of the distance between $\Adv_{E,\lambda}^{prp}(D)$ and $\Adv_{E,\lambda}^{prf}(D)$. Before that, we need some preliminaries about birthday attack.

\section{Birthday Attack}
Let $S$ be a set of $N$ elements. We independently sample $q$ values $x_1, x_2, \cdots, x_q \getsr S$ at uniformly random. We will show that the collision probability $\Pr[\exists i\not= j : x_i = x_j]$ is roughly $\frac{q^2}{2N}$, i.e., about $1/2$ when $q$ is around $\sqrt{N}$.

In fact $\frac{q^2}{2N}$ is the upper bound of the collision probability. To give this bound, we need the union bound of probabilities.

\begin{fact}
{\bf Union Bound}. For any two events $E_1$, $E_2$, the probability of their union has an upper bound:
$$\Pr[E_1 \vee E_2] \leq \Pr[E_1] + \Pr[E_2].$$
\end{fact}

By this fact, we can show the upper bound.

\begin{lemma}
$\Pr[\exists i\not= j : x_i = x_j] \leq \frac{q^2}{2N}$.
\end{lemma}
\begin{proof}
We can rewrite the event as:
$$\Pr[\exists i\not= j : x_i = x_j] = \Pr[\bigvee_{1 \leq i < j \leq q} : x_i = x_j].$$
By using union bound of probabilities, we have
$$\Pr[\bigvee_{1 \leq i < j \leq q} : x_i = x_j] \leq \sum_{1\leq i < j \leq q}\Pr[x_i=x_j] = \sum_{1\leq i < j \leq q} \frac{1}{N} = \binom{q}{2} \frac{1}{N} \leq \frac{q^2}{2N}.$$
\end{proof}

\section{PRP v.s. PRF}
By birthday attack, we can build a more efficient distinguisher to tell a block cipher apart from random function. 

Given any block cipher $E : \bits^\lambda \times \bits^n \to \bits^n$, we build a distinguisher $D$ that queries around $\sqrt{2^n} = 2^{n/2}$ distinct inputs $x_1, x_2, \cdots, x_{2^{n/2}}$. The corresponding outputs are $y_1, y_2, \cdots, y_{2^{n/2}}$. $D$ outputs 1 if there is no collision among $y_1, y_2, \cdots, y_{2^{n/2}}$, and outputs 0 otherwise.

Clearly, $\Pr[s \getsr \bits^\lambda : D^{E(s, \cdot)} = 1] = 1$, and by the upper bound of birthday attack
$$\Pr[D^{RF_{n,n}}=1] \geq 1-\frac{(\sqrt{2^n})^2}{2\cdot 2^n} = \frac{1}{2}.$$
It turns out the advantage $\Adv_{E,\lambda}^{prf}(D)$ is roughly around some non-zero constant. This means if $n$ is relatively small, we can efficiently distinguish the block cipher from a random function. For example, in DES we have $n=64$. Then $2^{n/2} = 2^{32}$ is very small for modern computers. In this sense PRF may not be a good notion for secure block ciphers.

The following theorem puts the above discussion in a clear way, i.e., given sufficiently large $n$, the distinguisher advantages up to PRP and PRF respectively will be very close.
\begin{theorem}
Given a block cipher $E : \bits^\lambda \times \bits^n \to \bits^n$, for all distinguisher $D$ making at most $q$ queries, 
$$\bigg| \Adv_{E,\lambda}^{prf}(D) - \Adv_{E,\lambda}^{prp}(D) \bigg| \leq \frac{q^2}{2 \cdot 2^n}.$$
\end{theorem}
\begin{proof}
Define
$$
\Delta \eqdef \bigg| \Adv_{E,\lambda}^{prf}(D) - \Adv_{E,\lambda}^{prp}(D) \bigg| 
= \bigg| \Pr[D^{RF_{n,n}}=1] - \Pr[D^{RP_n}=1] \bigg|.
$$
It suffices to show $\Delta \leq \frac{q^2}{2 \cdot 2^n}$. 

Note that for any $D$, $D$ makes at most $q$ queries. We define a new system $G$ s.t.
\begin{itemize}
\item Initially, $T(x) \gets \bot$ for all $x \in \bits^n$. $cnt \gets 0$.
\item For a given list of $q$ results: $X = (x_1, x_2, \cdots, x_q)$, upon query $x$, we define $G(X)$ as
\begin{quote}
$G(X)(x)$:
\begin{enumerate}
\item {\bf if} $T[x] = \bot$ {\bf then}
\item \tab $cnt \gets cnt + 1$.
\item \tab $T[x] \gets X[cnt]$.
\item {\bf return} $T[x]$.
\end{enumerate}
\end{quote}
\end{itemize}

Define
$$A \eqdef \{(x_1,x_2,\cdots,x_q) | x_1,x_2,\cdots,x_q \in \bits^n \},$$
$$S \eqdef \{(x_1,x_2,\cdots,x_q) \in A | \forall i \not= j, x_i \not= x_j \}. $$
Clearly $S\subseteq A$. \footnote{For example, let $n=2$, $q=4$. $(00, 10, 11, 00)$ is in $A$ but not in $S$. $(00, 10, 11, 01)$ is in $S$, which is a subset of $A$.}

By the definition of $G$ above, we have
$$\Pr[X \getsr A : D^{G(X)}=1] = \Pr[D^{RF_{n,n}}=1],$$
$$\Pr[X \getsr S : D^{G(X)}=1] = \Pr[D^{RP_n}=1].$$
Thus
$$\begin{aligned}
\Delta = & \bigg| \Pr[X \getsr A : D^{G(X)}=1] - \Pr[X \getsr S : D^{G(X)}=1] \bigg| \\
=& \bigg| \Pr[X \getsr A : D^{G(X)}=1 \wedge X \in S] + \Pr[X \getsr A : D^{G(X)}=1 \wedge X \not\in S] \\
& - \Pr[X \getsr S : D^{G(X)}=1] \bigg|.
\end{aligned}$$
We have two facts about the probabilities.
\begin{fact}
$\Pr [X \getsr A : D^{G(X)}=1 \wedge X \not\in S] \leq \Pr [X \getsr A : X \not\in S] \leq \frac{q^2}{2 \cdot 2^n}$. The last inequality is given by birthday attack.
\end{fact}

\begin{fact}
$$\begin{aligned}
& \Pr[X \getsr A : D^{G(X)}=1 \wedge X \in S] \\
= & \Pr[X \getsr A : X \in S] \Pr[X \getsr S : D^{G(X)}=1] \\
\leq &\Pr[X \getsr S : D^{G(X)}=1].
\end{aligned}$$
This is by the law of conditional probability.
\end{fact}
Both facts give $\Delta \leq \frac{q^2}{2 \cdot 2^n}$.
\end{proof}

\section{PRP Constructions}
Given a PRF $F : \bits^\lambda \times \bits^n \to \bits^n$, how to build a block cipher $E : \bits^\lambda \times \bits^{n'} \to \bits^{n'}$ which is a secure PRP?

Feistel construction can achieve this goal. In fact, it is an idea behind DES.

\begin{quote}
Feistel construction: build a block cipher $E : \bits^{r\lambda} \times \bits^{2n} \to \bits^{2n}$, where $r$ is the number of rounds.

Procedure $E((s_1, s_2, \cdots, s_r), (x_0, x_1))$ ($s_i\in\bits^\lambda$, $x_i\in\bits^n$)
\begin{enumerate}
\item {\bf for} $i=1$ to $r$ {\bf do}
\item \tab $x_{i+1} \gets F(s_i, x_i) \oplus x_{i-1}$.
\item {\bf return} $(x_r, x_{r+1})$.
\end{enumerate}
\end{quote}

Luby and Rackoff \cite{LR88} showed that if the number of rounds is no less than 3, then the Feistel construction gives a PRP block cipher.
\begin{theorem}
For any block cipher $E$ built by $r$-round Feistel construction, if $r\geq 3$ and $F$ is a PRF, then $E$ is a PRP.
\end{theorem}

\begin{thebibliography}{10}

\bibitem{LR88}
M. Luby and C. Rackoff, 1988. 
How to construct pseudorandom permutations from pseudorandom functions. 
SIAM Journal on Computing, 17(2), pp.373-386.

\end{thebibliography}

\end{document}
