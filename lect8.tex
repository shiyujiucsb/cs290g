\documentclass[12pt]{article}
\usepackage{url,amsmath,setspace,amssymb,amsthm,amsfonts}
%\usepackage{hyperref}


\setlength{\oddsidemargin}{.25in}
\setlength{\evensidemargin}{.25in}
\setlength{\textwidth}{6.25in}
\setlength{\topmargin}{-0.4in}
\setlength{\textheight}{8.5in}

\newcommand{\heading}[5]{
   \renewcommand{\thepage}{#1-\arabic{page}}
   \noindent
   \begin{center}
   \framebox[\textwidth]{
     \begin{minipage}{0.9\textwidth} \onehalfspacing
       {\bf CS 290G -- Introduction to Modern Cryptography} \hfill #2

       {\centering \Large #5
       
       }\medskip

       {\it #3 \hfill #4}
     \end{minipage}
   }
   \end{center}
}

\newcommand{\handout}[3]{\heading{#1}{#2}{Instructor:
Stefano Tessaro}{Scribe: Shiyu Ji}{Lecture #1: #3}}

\setlength{\parindent}{0in}

\newcommand{\eqdef}{\stackrel{def}{=}}
\newcommand{\N}{\mathbb{N}}
\newcommand{\R}{\mathbb{R}}
\newcommand{\Z}{\mathbb{Z}}
\newcommand{\F}{\mathbb{F}}
\newcommand{\bits}{\{0,1\}}
\newcommand{\inr}{\in_{\mbox{\tiny R}}}
%\newcommand{\getsr}{\gets_{\mbox{\tiny R}}}
\newcommand{\getsr}{\stackrel{\$}{\gets}}
\newcommand{\st}{\mbox{ s.t. }}
\newcommand{\etal}{{\it et al }}
\newcommand{\into}{\rightarrow}

\newcommand{\Ex}{\mathbb{E}}
\newcommand{\e}{\epsilon}
\newcommand{\ee}{\varepsilon}
\newcommand{\ceil}[1]{{\lceil{#1}\rceil}}
\newcommand{\floor}[1]{{\lfloor{#1}\rfloor}}
\newcommand{\angles}[1]{\langle #1 \rangle}
\newcommand{\Com}{{\sf Com}}
\newcommand{\desc}{{\sf desc}}

\newcommand{\rightstep}[1]{%
$\underrightarrow{\quad #1 \quad}$ }

\newcommand{\leftstep}[1]{%
$\underleftarrow{\quad #1 \quad}$ }

\newcommand{\Adv}{\mathsf{Adv}}
\newcommand{\poly}{\mathsf{poly}}
\newcommand{\MAC}{\mathsf{MAC}}
\newcommand{\Ver}{\mathsf{Ver}}

\newcommand{\tab}{\hspace{0.3in}}
%%%%%%%%%%%%%%%%%%%%%%%%%%%%
% Theorems & Definitions


\newtheorem{theorem}{Theorem}[section]

\newtheorem{claim}[theorem]{Claim}
\newtheorem{subclaim}{Claim}[theorem]
\newtheorem{proposition}[theorem]{Proposition}
\newtheorem{lemma}[theorem]{Lemma}
\newtheorem{corollary}[theorem]{Corollary}
\newtheorem{conjecture}[theorem]{Conjecture}

\theoremstyle{definition}
\newtheorem{definition}[theorem]{Definition}
\newtheorem{construction}[theorem]{Construction}
\newtheorem{example}[theorem]{Example}
\newtheorem{algorithm1}[theorem]{Algorithm}
\newtheorem{protocol}[theorem]{Protocol}
\newtheorem{remark}[theorem]{Remark}
\newtheorem{observation}[theorem]{Observation}
\newtheorem{assumption}[theorem]{Assumption}
\newtheorem{fact}[theorem]{Fact}

%\bibliographystyle{plain}
\usepackage{tikz}
\usetikzlibrary{calc,decorations.pathreplacing}

\begin{document}
\handout{8}{Feb 2, 2016}{Message Authentication}
\section{Message Authentication Code}
Last lecture we saw symmetric encryption, in which the adversary Eve can wiretap the channel between Alice and Bob. In this lecture we consider a more powerful adversary Eve, who can tamper with or even forge the messages sent from Alice to Bob. We can use \emph{message authentication code} (MAC) to defend against such an adversary Eve.
Message authentication code (MAC) is an efficient algorithm $\MAC : \bits^\lambda \times \bits^* \to \bits^n$ (typically $n \approx \lambda$). 
Alice uses $\MAC$ to generate a tag $\tau \in \bits^n$ for the message $m\in\bits^*$ by using the key $k\in\bits^\lambda$: $\tau \gets \MAC(k, m)$. Then Alice sends the pair $(m, \tau)$ to Bob. By wiretapping the channel, Eve obtains $(m, \tau)$ and may change it to $(m', \tau')$ and then sends $(m', \tau')$ to Bob. After receiving $(m', \tau')$, Bob verifies the tag $\tau'$ by using $\MAC$ and the shared key $k$ as following:
\begin{quote}
Bob's verification $\Ver(k, m', \tau')$:
\begin{enumerate}
\item $\tilde{\tau} \gets \MAC(k, m')$.
\item {\bf if} $\tilde{\tau} = \tau'$ {\bf then}
\item \tab {\bf output} $m'$.
\item {\bf else output} $\bot$.
\end{enumerate}
\end{quote}
The following graph shows the model here:
\begin{quote}
\begin{tikzpicture}
\draw (0,0) rectangle (2,1); \node at (1,0.5) {Alice};
\node at (1,-0.5) {$\tau \gets \MAC(k, m)$};
\draw [->] (2,0.5) -- (4,0.5);
\node [above] at (3,0.5) {$(m, \tau)$};
\draw (4,0) rectangle (6,1); \node at (5,0.5) {Eve}; 
\draw [->] (6,0.5) -- (8,0.5);
\node [above] at (7,0.5) {$(m', \tau')$};
\draw (8,0) rectangle (10,1); \node at (9,0.5) {Bob};
\node at (9,-0.5) {$\Ver(k, m', \tau')$};
\draw [->] (10,0.5) -- (11,0.5);
\node at (12,0.5) {$m$ or $\bot$};
\end{tikzpicture}
\end{quote}

We now discuss what a secure MAC should be. Informally, any adversary Eve cannot tamper with or forge any message or tag, even with chosen-message attack. 
That is, a secure MAC should have unforgeability under chosen-message attack (UF-CMA).

\begin{definition}
A MAC is secure if for all PPT adversary $A$, the advantage
$$\begin{aligned}
\Adv_{\MAC}^{uf-cma}(A) \eqdef & \Pr[s \getsr \bits^\lambda, (m, \tau) \gets A^{\MAC(s,\cdot)} : \\
& \MAC(s,m)=\tau \wedge \textrm{$m$ was not queried to $\MAC(s,\cdot)$ by $A$}]
\end{aligned}$$
is negligible.
\end{definition}

How to build a secure MAC? In fact, there are already many MACs in practice, e.g., HMAC, CBC-MAC, etc.

We will show that if there exists a PRF that can be extended to work with inputs of any length (i.e. inputs from $\bits^*$), then we can construct a MAC. We first extend the definition of PRF. Recall that in the definition of PRF, the advantage is defined as
$$\Adv_{F,\lambda}^{prf}(D) \eqdef \bigg| \Pr[s \getsr \bits^\lambda: D^{F(s,\cdot)}=1] - \Pr[D^{RF_{m,n}}=1] \bigg|.$$
To extend the definition to work with $\bits^*$, we can replace $RF_{m,n}$ with $RF_{*,n}$, which can take inputs from $\bits^*$.

We next give the construction of MAC as the following theorem.

\begin{theorem}
If $F : \bits^\lambda \times \bits^* \to \bits^n$ is a PRF, and $n = \omega(\log \lambda)$,\footnote{Equivalently, $2^{-n}$ is negligible.} then $F$ is a secure MAC.
\end{theorem}
\begin{proof}
It suffices to show that for all adversary $A$, there exists a distinguisher $D$ s.t.
$$\Adv_{\MAC}^{uf-cma}(A) \leq \Adv_{F}^{prf}(D) + \frac{1}{2^n}.$$
Moreover, if $A$ is PPT, then $D$ is also PPT. The construction of $D$ is given as following:
\begin{quote}
Distinguisher $D^O$: // $O$ can be either $F(s,\cdot)$ or $RF_{*,n}$.
\begin{enumerate}
\item $(m,\tau) \getsr A^O$.
\item {\bf if} $O(m) = \tau \wedge$ $m$ is not queried to $O$ by $A$ {\bf then}
\item \tab {\bf return} 1.
\item {\bf else return} 0.
\end{enumerate}
\end{quote}
Clearly
$$\begin{aligned}
\Pr[s\getsr\bits^\lambda: D^{F(s,\cdot)}=1] =& \Pr[s\getsr\bits^\lambda, (m,\tau) \getsr A^{F(s,\cdot)}: F(s,m) = \tau \\
&\wedge \textrm{$m$ was not queired to $F(s,\cdot)$ by $A$}] = \Adv_{F}^{uf-cma}(A).
\end{aligned}$$
Hence we can rewrite $\Adv_{F}^{uf-cma}(A)$ as
$$\begin{aligned}
&\Adv_{F}^{uf-cma}(A) 
= \Pr[s\getsr\bits^\lambda: D^{F(s,\cdot)}=1] - \Pr[D^{RF_{*,n}}=1] + \Pr[D^{RF_{*,n}}=1].
\end{aligned}$$
By 
$$\Adv_{F}^{prf}(D) \eqdef \bigg| \Pr[s\getsr\bits^\lambda: D^{F(s,\cdot)}=1] - \Pr[D^{RF_{*,n}}=1] \bigg|,$$
we have
$$\Pr[s\getsr\bits^\lambda: D^{F(s,\cdot)}=1] - \Pr[D^{RF_{*,n}}=1] \leq \Adv_{F}^{prf}(D).$$
The remaining is to bound $\Pr[D^{RF_{*,n}}=1]$.
\begin{claim}
$\Pr[D^{RF_{*,n}}=1] \leq 2^{-n}$.
\end{claim}
{\bf Proof of the claim}: If $D$ outputs 1, then in the if-statement, $m$ must be queried to $RF_{*,n}$ for the first time, after $\tau$ has been fixed. This implies $\Pr[RF_{*,n}(m) = \tau] = 2^{-n}$.
\end{proof}
The theorem above shows that we can construct MAC from PRF. Does the theorem hold conversely? That is, is every MAC a PRF? 
The answer is no. Consider $\MAC(s,m) = F(s,m)||0$, where $F$ is a PRF. It turns out this MAC is valid, but not a PRF.

\section{Almost-Universal Hash Function}
Almost-Universal Hash Function (AUHF) is important in the sense that together with PRF (with fixed input length), we can construct PRF with inputs in $\bits^*$.

\begin{definition}
A function $H : \bits^\lambda \times \bits^* \to \bits^n$ is {\bf $\delta$-almost-universal hash function} ($\delta$-AUHF) for some function $\delta : \N \to [0,1]$, \footnote{For example, $\delta(\ell) = \ell/2^n$.} if for all $m \not= m'$,
$$\Pr[s \getsr \bits^\lambda : H(s,m) = H(s, m')] \leq \delta(\max\{|m|, |m'|\}).$$ 
\end{definition}

We can concretely construct an AUHF. Before that we need some preliminaries about fields. Very informally, a field $\F$ is a set equipped with addition and multiplication, and we always have inverses for both of them. For example, for any prime $p$, $\Z_p$ is a field taking addition and multiplication modulo $p$. $\Z_2^n = \bits^n$ is another field with modular component-wise addition (aka $\oplus$) and multiplication. 

Now given a field $\F$, we can define $H : \F \times \F^* \to \F$ as
$$H(s, m=(m_1, m_2, \cdots, m_\ell)) \eqdef \sum_{i=1}^{\ell} m_i \cdot s^{i-1}.$$
We assume $m = (m_1, m_2, \cdots, m_\ell)$ and $m' = (m'_1, m'_2, \cdots, m'_\ell)$ have the same length. We now argue that $H$ is an AUHF.

\begin{lemma}
The above $H$ is $\frac{\ell}{|\F|}$-AUHF.
\end{lemma}
\begin{proof}
We may directly compute the probability:
$$\begin{aligned}
& \Pr[s \getsr \F : H(s,m) = H(s,m')] \\ 
=& \Pr[s \getsr \F : \sum_{i=1}^{\ell} m_i \cdot s^{i-1} = \sum_{i=1}^{\ell} m'_i \cdot s^{i-1}] \\
=& \Pr[s \getsr \F : \sum_{i=1}^{\ell} (m_i-m'_i) \cdot s^{i-1} = 0].
\end{aligned}$$
Since $m \not= m'$, there exists some $j \leq \ell$ s.t. $m_j \not= m'_j$. That is, $s$ must be a root of a polynomial with degree at most $\ell-1$. By Schwartz–Zippel lemma, the number of the roots is at most $\ell-1$. Thus
$$\Pr[s \getsr \F : H(s,m) = H(s,m')] \leq \frac{\ell-1}{|\F|}.$$
One can take $\delta = \frac{\ell}{|\F|}$.
\end{proof}

\section{Construction of MAC}
Now we will show how to construct MAC by using PRF and AUHF. Given a PRF $F : \bits^\lambda \times \bits^n \to \bits^n$ and an AUHF $H : \bits^\lambda \times \bits^* \to \bits^n$, we can construct a $\MAC : \bits^{2\lambda} \times \bits^* \to \bits^n$ as following:
$$\MAC (s = (s_1,s_2), m) \eqdef F(s_2, H(s_1, m)).$$
The following theorem shows that the above $\MAC$ is valid.

\begin{theorem}
If $F$ is a PRF, and $H$ is $\delta$-AUHF ($\delta$ is negligible), then the above $\MAC$ is a PRF.
\end{theorem}
\begin{proof}
It suffices to show that for any distinguisher $D$, there exists another distinguisher $D'$ s.t. 
$$\Adv_{\MAC}^{prf}(D) \leq \Adv_{\MAC}^{prf}(D') + q^2 \delta(\ell),$$
if $D$ makes at most $q$ queries at length at most $\ell$. 
Moreover, PPT $D$ implies PPT $D'$.
Note that if $D$ is PPT, then $\ell$ and $q$ are polynomial, and thus $q^2 \delta(\ell)$ is negligible.

To show this, we may replace $F(s_2,\cdot)$ with $RF_{n,n}$. That is
$$\begin{aligned}
&\Adv_{\MAC}^{prf}(D) = \bigg| \Pr[s_1,s_2 \getsr \bits^\lambda : D^{\MAC((s_1,s_2),\cdot)}=1] - \Pr[D^{RF_{*,n}}=1] \bigg| \\
\leq & \bigg| \Pr[s_1,s_2 \getsr \bits^\lambda : D^{\MAC((s_1,s_2),\cdot)}=1] - \Pr[s_1 \getsr \bits^\lambda : D^{RF_{n,n}(H(s_1,\cdot))}=1] \bigg| \\
&+ \bigg| \Pr[s_1 \getsr \bits^\lambda : D^{RF_{n,n}(H(s_1,\cdot))}=1] - \Pr[D^{RF_{*,n}}=1] \bigg|. \\
\end{aligned}$$
Define
$$\Delta_1 \eqdef \bigg| \Pr[s_1,s_2 \getsr \bits^\lambda : D^{\MAC((s_1,s_2),\cdot)}=1] - \Pr[s_1 \getsr \bits^\lambda : D^{RF_{n,n}(H(s_1,\cdot))}=1] \bigg|.$$
$$\Delta_2 \eqdef \bigg| \Pr[s_1 \getsr \bits^\lambda : D^{RF_{n,n}(H(s_1,\cdot))}=1] - \Pr[D^{RF_{*,n}}=1] \bigg|.$$
Note that by the construction of $\MAC$,
$$\Delta_1 = \bigg| \Pr[s_1,s_2 \getsr \bits^\lambda : D^{F(s_2,H(s_1,\cdot))}=1] - \Pr[s_1 \getsr \bits^\lambda : D^{RF_{n,n}(H(s_1,\cdot))}=1] \bigg|.$$
We first bound $\Delta_1$. This can be done by constructing the following distinguisher $D'$.
\begin{quote}
Distinguisher $D'^O$: // $O$ can be either $F(s_2, \cdot)$ or $RF_{n,n}$.
\begin{enumerate}
\item $s_1 \getsr \bits^\lambda$.
\item $b \getsr D^{O(H(s_1,\cdot))}$.
\item {\bf return} $b$.
\end{enumerate}
\end{quote}
Clearly $\Delta_1 = \Adv_{F}^{prf}(D')$.

We next bound $\Delta_2$. First define two oracles:
\begin{quote}
\begin{minipage}{.3\textwidth}
$O_{s_1}(m)$:
\begin{enumerate}
\item $y \gets H(s_1, m)$.
\item $z \gets RF_{n,n}(y)$.
\item {\bf return} $z$.
\end{enumerate}
\end{minipage}
\begin{minipage}{.3\textwidth}
$O'_{s_1}(m)$:
\begin{enumerate}
\item $y \gets H(s_1, m)$.
\item $z \gets RF_{*,n}(m)$.
\item {\bf return} $z$.
\end{enumerate}
\end{minipage}
\end{quote}
Then 
$$\Delta_2 = \bigg| \Pr[s_1 \getsr \bits^\lambda : D^{O_{s_1}(\cdot)}=1] - \Pr[s_1 \getsr \bits^\lambda : D^{O'_{s_1}(\cdot)}=1] \bigg|.$$
For both $O_{s_1}(m)$ and $O'_{s_1}(m)$, define event $COLL$: there exists a collision among the $y$ values.
Since up to the first collision, the two oracles are identical. That is,
$$\Pr[s_1 \getsr \bits^\lambda : D^{O_{s_1}(\cdot)}=1 \wedge \neg COLL] = \Pr[s_1 \getsr \bits^\lambda : D^{O'_{s_1}(\cdot)}=1 \wedge \neg COLL].$$
Then we have 
$$\begin{aligned}
& \Delta_2 \\
 =& \bigg| \Pr[s_1 \getsr \bits^\lambda : D^{O_{s_1}(\cdot)}=1 \wedge COLL] - \Pr[s_1 \getsr \bits^\lambda : D^{O'_{s_1}(\cdot)}=1 \wedge COLL] \bigg| \\
\leq& \max\{\Pr[s_1 \getsr \bits^\lambda : D^{O_{s_1}(\cdot)}=1 \wedge COLL], \Pr[s_1 \getsr \bits^\lambda : D^{O'_{s_1}(\cdot)}=1 \wedge COLL]\} \\
\leq& \Pr[COLL \textrm{ when $D$ interacts with $O_{s_1}(\cdot)$ or $O'_{s_1}(\cdot)$}].
\end{aligned}$$
We claim that for any fixed $m_1,m_2,\cdots,m_q$, if $H$ is $\delta$-AUHF, then 
$$\Pr[s_1 \getsr \bits^\lambda : \exists i \not= j, H(s_1,m_i) = H(s_1,m_j)] \leq q^2\delta(\ell).$$
The argument is birthday attack. 
The details will be given next lecture.
\end{proof}

\end{document}
