\documentclass[12pt]{article}
\usepackage{url,amsmath,setspace,amssymb,amsthm,amsfonts}
%\usepackage{hyperref}

\setlength{\oddsidemargin}{.25in}
\setlength{\evensidemargin}{.25in}
\setlength{\textwidth}{6.25in}
\setlength{\topmargin}{-0.4in}
\setlength{\textheight}{8.5in}

\newcommand{\heading}[5]{
   \renewcommand{\thepage}{#1-\arabic{page}}
   \noindent
   \begin{center}
   \framebox[\textwidth]{
     \begin{minipage}{0.9\textwidth} \onehalfspacing
       {\bf CS 290G -- Introduction to Modern Cryptography} \hfill #2

       {\centering \Large #5
       
       }\medskip

       {\it #3 \hfill #4}
     \end{minipage}
   }
   \end{center}
}

\newcommand{\handout}[3]{\heading{#1}{#2}{Instructor:
Stefano Tessaro}{Student: Shiyu Ji, Yi Yang}{#3}}

\setlength{\parindent}{0in}

\newcommand{\eqdef}{\stackrel{def}{=}}
\newcommand{\N}{\mathbb{N}}
\newcommand{\R}{\mathbb{R}}
\newcommand{\Z}{\mathbb{Z}}
\newcommand{\F}{\mathbb{F}}
\newcommand{\bits}{\{0,1\}}
\newcommand{\inr}{\in_{\mbox{\tiny R}}}
%\newcommand{\getsr}{\gets_{\mbox{\tiny R}}}
\newcommand{\getsr}{\stackrel{\$}{\gets}}
\newcommand{\getsn}{\stackrel{n}{\gets}}
\newcommand{\st}{\mbox{ s.t. }}
\newcommand{\etal}{{\it et al }}
\newcommand{\into}{\rightarrow}

\newcommand{\Ex}{\mathbb{E}}
\newcommand{\e}{\epsilon}
\newcommand{\ee}{\varepsilon}
\newcommand{\ceil}[1]{{\lceil{#1}\rceil}}
\newcommand{\floor}[1]{{\lfloor{#1}\rfloor}}
\newcommand{\angles}[1]{\langle #1 \rangle}
\newcommand{\Com}{{\sf Com}}
\newcommand{\desc}{{\sf desc}}

\newcommand{\rightstep}[1]{%
$\underrightarrow{\quad #1 \quad}$ }

\newcommand{\leftstep}[1]{%
$\underleftarrow{\quad #1 \quad}$ }
\newcommand{\Adv}{\textsf{Adv}}
\newcommand{\tab}{\hspace{0.3in}}
\newcommand{\MAC}{\textsf{MAC}}
\newcommand{\CBC}{\textsf{CBC}}
\newcommand{\Eval}{\textsf{Eval}}
\newcommand{\Vrfy}{\textsf{Vrfy}}
\newcommand{\WIN}{\textsf{WIN}}
\newcommand{\true}{\textsf{true}}
\newcommand{\false}{\textsf{false}}
\newcommand{\M}{\mathcal{M}}
\newcommand{\C}{\mathcal{C}}

%%%%%%%%%%%%%%%%%%%%%%%%%%%%
% Theorems & Definitions


\newtheorem{theorem}{Theorem}[section]

\newtheorem{claim}[theorem]{Claim}
\newtheorem{subclaim}{Claim}[theorem]
\newtheorem{proposition}[theorem]{Proposition}
\newtheorem{lemma}[theorem]{Lemma}
\newtheorem{corollary}[theorem]{Corollary}
\newtheorem{conjecture}[theorem]{Conjecture}
\newtheorem{observation}[theorem]{Observation}

\theoremstyle{definition}
\newtheorem{definition}[theorem]{Definition}
\newtheorem{construction}[theorem]{Construction}
\newtheorem{example}[theorem]{Example}
\newtheorem{counterexample}[theorem]{Counterexample}
\newtheorem{algorithm1}[theorem]{Algorithm}
\newtheorem{protocol}[theorem]{Protocol}
\newtheorem{remark}[theorem]{Remark}
\newtheorem{assumption}[theorem]{Assumption}
\newtheorem{fact}[theorem]{Fact}

%\bibliographystyle{plain}

\begin{document}
\handout{3}{Due: Feb 26, 2016}{Homework 3}
\section{Task 1 - CBC-MAC}
We define a variant of CBC-MAC as the following function $\CBC : \bits^\lambda \times \bits^\lambda \to \bits^n$ using a block cipher $E : \bits^\lambda \times \bits^n \to \bits^n$:
\begin{quote}
{\bf Procedure} $\CBC_k (M)$: // $M \in \bits^*$
\begin{enumerate}
\item $M[1], \cdots, M[\ell] \getsn M$
\item $S_0 \gets 0^n$
\item {\bf for} $i=1$ to $\ell$ {\bf do} $S_i \gets E_k (S_{i-1} \oplus M[i])$
\item {\bf return} $S_{\ell}$
\end{enumerate}
\end{quote}
Recall that the $\getsn$ padding operation first appends to $M$ a single 1 bit and then appends as many 0's as necessary to make the length of $M$ a multiple of $n$ (say $\ell \cdot n$). Finally, it outputs the $\ell$ $n$-bit blocks of the extension of $M$.

{\bf a) [Points: 8]} Prove that the $\CBC$ construction given above is \emph{not} UF-CMA secure. In particular, show first that given $Y = \CBC_k (M)$ for some message $M$ and (unknown) key $k$, there exist two other messages $M', M''$ s.t. $M'' \not\in \{M, M'\}$ with the following property:
\begin{quote}
Using $Y$ and the result $Y' = \CBC_k (M')$ of the query $M'$ to $\CBC_k$, we can compute $Y'' = \CBC_k (M'')$ without querying $M$.
\end{quote}
Use this to describe a concrete attacker $A$ making two queries!

\begin{proof}
We directly give the adversary, and then observe the $M$, $M'$, $Y$ and $Y'$ that are used.
\begin{quote}
{\bf Adversary} $A^{\Eval, \Vrfy}$:
\begin{enumerate}
\item $\WIN \gets \false$
\item $M \gets 0^n$
\item $Y \gets \Eval(M)$
\item $M' \gets 1^n$
\item $Y' \gets \Eval(M')$
\item Query $\Vrfy(0^n10^{n-1}||(Y \oplus 1^n), Y')$
\end{enumerate}
\end{quote}
Note that 
$$Y = \CBC_k (0^n) = E_k(E_k(0^n) \oplus 10^{n-1}),$$
$$Y' = \CBC_k (1^n) = E_k(E_k(1^n) \oplus 10^{n-1}).$$
Hence
$$\begin{aligned}
Y' &= E_k(E_k(Y \oplus Y \oplus 1^n) \oplus 10^{n-1}) \\
&= E_k(E_k(E_k(E_k(0^n) \oplus 10^{n-1}) \oplus Y \oplus 1^n) \oplus 10^{n-1}) \\
&= \CBC_k (0^n10^{n-1} || (Y \oplus 1^n)).
\end{aligned}$$
Thus $\WIN$ will always be set to $\true$ in the last $\Vrfy$. Hence we have $\Adv_{CBC}^{uf-cma}(A) = 1$.
\end{proof}
Note that the adversary $A$ can also query $\Vrfy(0^n10^{n-1}||Y, Y)$. The advantage is still 1. Hence only one query $\Eval(0^n)$ would be enough.

{\bf b) [Points: 4]} How can we modify the encoding of $M$ into blocks
$M[1], \cdots, M[\ell]$ so that the attack from {\bf a)} is not possible any more?

{\bf Solution}: Note that in {\bf a)} we query two messages: $0^n$ and $0^n10^{n-1}||(Y \oplus 1^n)$. $0^n$ is a proper prefix of the other. If we require any queried message cannot be a prefix of any other one, such attack will fail. As suggested by \cite{BKR94}, assuming $\ell < 2^n$, this can be done by prepending the block length $\ell$ before $M[1], \cdots, M[\ell]$.
\begin{quote}
{\bf Procedure} $\CBC_k (M)$: // $M \in bits^*$
\begin{enumerate}
\item $M[1], \cdots, M[\ell] \getsn M$
\item $M[0] \gets \ell$
\item $S_{-1} \gets 0^n$
\item {\bf for} $i=0$ to $\ell$ {\bf do} $S_i \gets E_k (S_{i-1} \oplus M[i])$
\item {\bf return} $S_{\ell}$
\end{enumerate}
\end{quote}
There may be some other ways to achieve the goal that no queried message is a prefix of another. But we still need to prove their security.


\section{Task 2 - CCA-Security and Authenticated Encryption}
Let $SE = (Kg, Enc, Dec)$ be a symmetric encryption scheme. Further, assume that $SE$ is INT-PTXT and IND-CPA secure. We build a new encryption scheme $SE' = (Kg', Enc', Dec')$ such that $Kg' = Kg$ and for all valid keys $k$, messages $M$ and ciphertexts $C$:
\begin{quote}
\begin{minipage}[t]{0.3\textwidth}
{\bf Procedure} $Enc'_k(M)$:
\begin{enumerate}
\item $C' \getsr Enc_k (M)$
\item {\bf Return} $C||0$
\end{enumerate}
\end{minipage}
\begin{minipage}[t]{0.5\textwidth}
{\bf Procedure} $Dec'_k(M)$:
\begin{enumerate}
\item Parse $C$ as $C'||b$, where $b \in \bits$
\item $M' \getsr Dec_k (C')$
\item {\bf Return} $M'$
\end{enumerate}
\end{minipage}
\end{quote}

{\bf a) [Points: 4]} Prove that $SE'$ is INT-PTXT secure. In particular, prove that for all adversary $A$, there exists an adversary $A'$ such that
$$\Adv_{SE'}^{int-ptxt}(A) = \Adv_{SE}^{int-ptxt}(A'),$$
and verify that $A'$ is PPT whenever $A$ is PPT.

\begin{proof}
The game-based INT-PTXT security of $SE'$ is defined as following.
\begin{quote}
\begin{minipage}[t]{0.35\textwidth}
{\bf Experiment} $Exp_{SE'}^{int-ptxt}(A)$:
\begin{enumerate}
\item $\WIN \gets \false$
\item $\M \gets \emptyset$
\item $k \getsr Kg'(1^\lambda)$
\item Run $A^{E,D}$
\end{enumerate}
\end{minipage}
\begin{minipage}[t]{0.5\textwidth}
{\bf Procedure} $E(M)$:
\begin{enumerate}
\item $C \getsr Enc'_k(M)$
\item $\M \gets \M \cup \{M\}$
\item {\bf return} $C$
\end{enumerate}
\end{minipage}

\begin{minipage}[t]{0.5\textwidth}
{\bf Procedure} $D(C)$:
\begin{enumerate}
\item $M \gets Dec'_k(C)$
\item {\bf if} $M \not\in \M$ and $M \not=\bot$ {\bf then}
\item \tab $\WIN \gets \true$
\item {\bf return} $\WIN$
\end{enumerate}
\end{minipage}
\end{quote}
The INT-PTXT advantage of $A$ is
$$\Adv_{SE'}^{int-ptxt}(A) = \Pr[\textrm{$\WIN = \true$ in $Exp_{SE'}^{int-ptxt}(A)$}].$$
Similarly we can define another adversary $A'$ that can perfectly simulate $A$.
\begin{quote}
\begin{minipage}[t]{0.35\textwidth}
{\bf Experiment} $Exp_{SE}^{int-ptxt}(A')$:
\begin{enumerate}
\item $\WIN \gets \false$
\item $\M \gets \emptyset$
\item $k \getsr Kg(1^\lambda)$
\item $A'$ runs $A^{E,D}$
\end{enumerate}
\end{minipage}
\begin{minipage}[t]{0.5\textwidth}
{\bf Procedure} $E(M)$:
\begin{enumerate}
\item $C \getsr Enc_k(M)$
\item $\M \gets \M \cup \{M\}$
\item {\bf return} $C||0$
\end{enumerate}
\end{minipage}

\begin{minipage}[t]{0.5\textwidth}
{\bf Procedure} $D(C)$:
\begin{enumerate}
\item Parse $C$ as $C'||b$, where $b \in \bits$
\item $M \gets Dec_k(C')$
\item {\bf if} $M \not\in \M$ and $M \not=\bot$ {\bf then}
\item \tab $\WIN \gets \true$
\item {\bf return} $\WIN$
\end{enumerate}
\end{minipage}
\end{quote}
Clearly $A'$ wins iff $A$ wins. Thus
$$\Adv_{SE'}^{int-ptxt}(A) = \Adv_{SE}^{int-ptxt}(A').$$
Since all the additional steps take polynomial time, $A'$ is PPT if $A$ is PPT.
\end{proof}

{\bf b) [Points: 4]} Prove that $SE'$ is not INT-CTXT secure.
\begin{proof}
The game-based INT-CTXT security of $SE'$ is defined as following.
\begin{quote}
\begin{minipage}[t]{0.35\textwidth}
{\bf Experiment} $Exp_{SE'}^{int-ctxt}(A)$:
\begin{enumerate}
\item $\WIN \gets \false$
\item $\C \gets \emptyset$
\item $k \getsr Kg'(1^\lambda)$
\item Run $A^{E,D}$
\end{enumerate}
\end{minipage}
\begin{minipage}[t]{0.5\textwidth}
{\bf Procedure} $E(M)$:
\begin{enumerate}
\item $C \getsr Enc'_k(M)$
\item $\C \gets \C \cup \{C\}$
\item {\bf return} $C$
\end{enumerate}
\end{minipage}

\begin{minipage}[t]{0.5\textwidth}
{\bf Procedure} $D(C)$:
\begin{enumerate}
\item $M \gets Dec'_k(C)$
\item {\bf if} $C \not\in \C$ and $M \not=\bot$ {\bf then}
\item \tab $\WIN \gets \true$
\item {\bf return} $\WIN$
\end{enumerate}
\end{minipage}
\end{quote}
Again the advantage of $A$ is the probability to win. We can construct a PPT adversary $A$ that can achieve noticeable advantage.
\begin{quote}
{\bf Adversary} $A^{E,D}$:
\begin{enumerate}
\item $C \gets E(0^\lambda)$
\item Parse $C$ as $C'||0$
\item Query $D(C'||1)$
\end{enumerate}
\end{quote}
Note that after the query $E(0^\lambda)$, $C'||0 \in \C$ but $C'||1 \not\in \C$. Since $C' \gets Enc_k(M)$, $M = Dec_k(C')$ cannot be $\bot$. Hence the if-condition in $D$ must be satisfied, and thus the advantage $\Adv_{SE'}^{int-ctxt}(A) = 1$.
\end{proof}

{\bf c) [Points: 2]} Prove that $SE'$ is not IND-CCA secure.
\begin{proof}
We give the adversary $A$ that can break IND-CCA security of $SE'$.
\begin{quote}
{\bf Adversary} $A^{O, D}$: // $O$ can be $LR_0^k$ or $LR_1^k$.
\begin{enumerate}
\item $C \gets O(0^\lambda, 1^\lambda)$
\item Parse $C$ as $C'||0$
\item $M \gets D(C'||1)$
\item {\bf if} $M=0^\lambda$ {\bf then return} 0.
\item {\bf else return} 1.
\end{enumerate}
\end{quote}
Following the same reasoning in {\bf b)}, we know that the message $M$ ($0^\lambda$ or $1^\lambda$) can always be restored, and $C'||1$ is never in $\C$. Thus $A$ can guess $b$ in $LR_b^k$ correctly with probability 1, i.e., $\Adv_{SE'}^{ind-cca}(A) = 1$.
\end{proof}

In the following, let additionally $\MAC : \bits^\lambda \times \bits^* \to \bits^n$ a MAC. We consider now the construction Encrypt-and-MAC discussed in class, which gives rise to the following symmetric encryption scheme $SE'' = (Kg'', Enc'', Dec'')$:
\begin{quote}
\begin{minipage}[t]{.3\textwidth}
{\bf Procedure} $Kg''(1^\lambda)$:
\begin{enumerate}
\item $k_{SE} \getsr Kg(1^\lambda)$
\item $k_{\MAC} \getsr \bits^\lambda$
\item {\bf Return} $k_{SE}||k_{\MAC}$
\end{enumerate}
\end{minipage}
\begin{minipage}[t]{.5\textwidth}
{\bf Procedure} $Enc''_{k_{SE}||k_{\MAC}}(M)$:
\begin{enumerate}
\item $C \getsr Enc_{k_{SE}} (M)$
\item $T \getsr \MAC_{k_{\MAC}}(M)$
\item {\bf Return} $C||T$
\end{enumerate}
\end{minipage}

\begin{minipage}[t]{.5\textwidth}
{\bf Procedure} $Dec''_{k_{SE}||k_{\MAC}}(C)$:
\begin{enumerate}
\item Parse $C$ as $C'||T$, where $T \in \bits^n$
\item $M' \gets Dec_{k_{SE}} (C')$
\item {\bf if} $\MAC_{k_{\MAC}}(M') = T$ {\bf then}
\item \tab {\bf Return} $M'$
\item {\bf Return} $\bot$
\end{enumerate}
\end{minipage}
\end{quote}

{\bf d) [Points: 6]} Prove that if $\MAC$ is UF-CMA secure, then $SE''$ is {\it never} IND-CPA secure, no matter how secure $SE$ is.
\begin{proof}
Assume $\MAC$ is deterministic. Denote by $k = k_{SE}||k_{\MAC}$. We have the following adversary $A$ that can break IND-CPA security of $SE''$.
\begin{quote}
{\bf Adversary} $A^O$: // $O$ can be $LR_0^k$ or $LR_1^k$.
\begin{enumerate}
\item $C_1 || T_1 \gets O(0^\lambda, 0^\lambda)$
\item $C_2 || T_2 \gets O(0^\lambda, 1^\lambda)$
\item {\bf if} $T_1 = T_2$ {\bf then return} 1.
\item {\bf else return} 0.
\end{enumerate}
\end{quote}
Note that if $O = LR_0^k$, then $T_1 = \MAC_{k_{\MAC}}(0^\lambda) = T_2$, and thus $A$ always returns 1. If $O = LR_1^k$, then $A$ returns 1 iff $T_1 = T_2$, i.e., $\MAC_{k_{\MAC}}(0^\lambda) = \MAC_{k_{\MAC}}(1^\lambda)$. Since $\MAC$ is UF-CMA secure, this happens with only negligible probability. Suppose an adversary $A'$ who guesses the messages $0^\lambda$ and $1^\lambda$ collide:
\begin{quote}
{\bf Adversary} $A'$:
\begin{enumerate}
\item $T \gets \textsf{Eval}(0^\lambda)$
\item Query $\textsf{Vrfy}(1^\lambda, T)$
\end{enumerate}
\end{quote}
We have
$$\Pr[T_1 = T_2] = \Pr[k \getsr Kg''(1^\lambda) : A^{LR_1^k}=1] = \Adv_{\MAC}^{uf-cma}(A'),$$
which is negligible. Putting both probabilities together, we have
$$\Adv_{SE''}^{ind-cpa}(A) = \bigg| \Pr[A^{LR_0^k}=1] -\Pr[A^{LR_0^k}=1] \bigg| = 1 - \Adv_{\MAC}^{uf-cma}(A'),$$
which is overwhelming.
\end{proof}

\section{Task 3 - Groups and Public-key Cryptography}

{\bf a) [Points: 2]} Consider the group $\Z_{10}$ of integers $\{0, 1, \cdots, 9\}$ with addition modulo 10. Is this group cyclic? Which elements are generators and which are not?

{\bf Solution}: Yes, $\Z_{10}$ is cyclic, and its generator can be any element coprime with 10, i.e., 1, 3, 7 and 9. 
\begin{claim}
Any integer coprime with $n$ can generate $\Z_n$.
\end{claim}
\begin{proof}
It suffices to show $i\cdot x \not= j \cdot x$ in $\Z_n$ for any $1\leq j < i \leq n$ and any $x$ coprime with $n$. Assume by contradiction $i\cdot x = j \cdot x \mod n$ for some $i > j$. Then $(i-j)\cdot x =0 \mod n$, i.e., there exists some integer $k$ s.t. $(i-j)\cdot x = k\cdot n$. Since $x$ is coprime with $n$, $(i-j)$ must be a multiple of $n$. However, $i-j < n$.
\end{proof}
Other elements in $\Z_{10}$ cannot generate the whole group, since 1 cannot be generated. Let $y$ be any element not coprime with $n$. Assume by contradiction there exists some integer $k$ s.t. $k\cdot y = 1 \mod n$, i.e., $k\cdot y + k' n = 1$ for some integer $k'$, which implies $y$ is coprime with $n$ by Theorem 1.8 in \cite{Shoup09}.

{\bf b) [Points: 3]} Consider the group $\Z_{17}^*$ of integers $\{1, \cdots, 16\}$ with multiplication modulo 17. Find a generator for $\Z_{17}^*$.

{\bf Solution}: By multiplication modulo 17, we have 
$$\angles{3} = \{3, 9, 10, 13, 5, 15, 11, 16, 14, 8, 7, 4, 12, 2, 6, 1\}.$$

{\bf c) [Points: 3]} Let $G$ be a group with $|G| = p$ elements, for a prime number $p\geq 2$. Prove that $G$ is a cyclic group.
\begin{proof}
We first need a claim.
\begin{claim}
Given any finite group $G$, for any element $x\in G$, its generated subset $\angles{x}$ is a subgroup of $G$.
\end{claim}
\begin{proof}
Denote by 1 the identity (aka neutral element) of $G$. Clearly the identity $x^0 = 1$. By pigeonhole principle, there exists a positive integer $k$ s.t. $x^k = 1$, since $G$ is finite. Hence the inverse for any element $x^i \in \angles{x}$ ($0\leq i \leq k-1$) can be found: $x^{k-i} \cdot x^i = x^{k} = 1$, i.e., $(x^i)^{-1} = x^{k-i}$. Hence $\angles{x}$ is a subgroup of $G$.
\end{proof}
By Lagrange's theorem, for any $x\in G$, $|\angles{x}|$ divides $|G|$. Since $|G|$ is a prime, $|\angles{x}|$ can only be 1 or $|G|$. If $x=1$ the identity of $G$, then $\angles{1} = \{1\}$. Otherwise, $|\angles{x}|>1$ and thus it must be $|\angles{x}| = |G|$, i.e., $x$ generates the whole $G$.

Lagrange's theorem can be found in \cite{Shoup09}.
\end{proof}

\section{Task 4 - The DDH Assumption}
Let $G = \angles{g}$ be a cyclic group. Recall that we define
$$\begin{aligned}
\Adv_G^{ddh}(D) \eqdef & \Pr[x_A, x_B \getsr \{0, 1, \cdots, |G|-1\} : D(g^{x_A}, g^{x_B}, g^{x_Ax_B})] \\
&- \Pr[x_A, x_B, x_C \getsr \{0, 1, \cdots, |G|-1\} : D(g^{x_A}, g^{x_B}, g^{x_C})],
\end{aligned}$$
and that the DDH-assumption is true for $G$ if $\Adv_G^{ddh}(D)$ is negligible for all PPT distinguishers $D$ (where polynomial refers to $\log |G|$).

{\bf a) [Points: 8]} Let $G = \Z_P^*$ (with multiplication modulo $P$) be a family of groups, where $P$ is a $\lambda$-bit prime, and let $g \in \Z_P^*$ be a generator.

Prove that the DDH assumption does \emph{not} hold in $G$. Even more precisely, prove that there exists a polynomial-time (in $\lambda$) distinguisher $D$ such that $\Adv_G^{ddh}(D) \geq 1/2$.

\begin{proof}
Even though there is no efficient way to find $x$ given $y = g^x$, we can compute $x \mod 2$ efficiently. Consider $y^{(P-1)/2} = g^{x(P-1)/2}$. If $x$ is even, then $y^{(P-1)/2} = g^{k(P-1)}$ for some integer $k$. Note that $g^{P-1}=1$ in $\Z_P^*$ by Fermat's little theorem. Thus $y^{(P-1)/2} = 1$ in $\Z_P^*$ if $x$ is even. If $x$ is odd, then in general $y^{(P-1)/2} = g^{x(P-1)/2}$ is not 1 in $\Z_P^*$ if $g^{(P-1)/2}\not=1 \mod P$, i.e., $P$ is not a safe prime.
This fact gives rise to an efficient distinguisher. 
We can give such a PPT distinguisher as following.
\begin{quote}
{\bf Distinguisher} $D (A, B, C)$: // $A = g^{x_A}$, $B = g^{x_B}$, $C = g^{x_C}$ or $g^{x_Ax_B}$
\begin{enumerate}
\item $a \gets A^{P-1} \mod P$
\item $b \gets B^{P-1} \mod P$
\item $c \gets C^{P-1} \mod P$
\item {\bf if} $(a=1 \vee b=1) \wedge c=1$ or none of $a, b, c$ is 1 {\bf then}
\item \tab {\bf return} 1.
\item {\bf else return} 0.
\end{enumerate}
\end{quote}
Assuming $P$ is not a safe prime, we have
$$\begin{aligned}
& \Adv_{G}^{ddh}(D) \\
\approx & \Pr[x_A, x_B \getsr \{0, 1, \cdots, |G|-1\} : (x_A\mod 2)\cdot (x_B \mod 2) = x_A\cdot x_B \mod 2] \\
&- \Pr[x_A, x_B, x_C \getsr \{0, 1, \cdots, |G|-1\} : (x_A\mod 2)\cdot (x_B \mod 2) = x_C \mod 2] \\
=& 1 - 1/2 = 1/2.
\end{aligned}$$
\end{proof}

{\bf b) [Points: 2]} Does the above imply that the CDH assumption or the DL assumption also do not hold in $G = G_\lambda$?

{\bf Solution}: No. Currently there is no known efficient way to solve CDH or DL over $G_\lambda$, e.g., $Z_P^*$ where $P$ is a $\lambda$-bit prime. If there is any implication, some efficient algorithms that can solve these long-standing problems immediately follow.

\section{Task 5 - One Query is Enough!}
Consider the definition of IND-CPA security for public-key encryption schemes. That is,
given a public-key encryption scheme $PKE = (Kg, Enc, Dec)$, we define $LR_b^k$ for $b\in\bits$ and a valid public key $pk$, which on input $M_0$, $M_1$ from the message space of $Enc(pk,\cdot)$, returns an encryption $C \getsr Enc(pk, M_b)$ of $M_b$. Then, we also define
$$\begin{aligned}
\Adv_{PKE}^{ind-cpa}(A) = \bigg| \Pr[(pk,sk)\getsr Kg(1^\lambda) : A^{LR_0^{pk}}(pk)=1] \\
-\Pr[(pk,sk)\getsr Kg(1^\lambda) : A^{LR_1^{pk}}(pk)=1] \bigg|
\end{aligned}$$
In this task, we want to prove that for all PPT $A$ making $q$ queries to the LR oracle, there exists a PPT $A'$ making \emph{one} query to the $LR$ oracle such that
\begin{equation}
\label{eq:1}
\Adv_{PKE}^{ind-cpa}(A) = q \cdot \Adv_{PKE}^{ind-cpa}(A').
\end{equation}
Concretely, solve the following sub-tasks:

{\bf a) [Points: 2]} For $i \in \{0,1,\cdots, q\}$, consider the oracle $\widetilde{LR}_i^{pk}$ which behaves as $LR_1^{pk}$ for the first $i$ queries, and then behaves as $LR_0^{pk}$ for the remaining $q-i$ queries. Prove that
$$\Adv_{PKE}^{ind-cpa}(A) = \bigg| \sum_{i=1}^q \Delta_i(A) \bigg|,$$
where
$$\begin{aligned}
\Delta_i(A) = \Pr[(pk,sk)\getsr Kg(1^\lambda) : A^{\widetilde{LR}_{i-1}^{pk}}(pk)=1] \\
- \Pr[(pk,sk)\getsr Kg(1^\lambda) : A^{\widetilde{LR}_{i}^{pk}}(pk)=1].
\end{aligned}$$
\begin{proof}
Define the hybrid oracle $\widetilde{LR}_i^{pk}$ as following: 
\begin{itemize}
\item For the first $i$ queries $(M_0,M_1)$, answer each of them with $Enc(pk, M_1)$.
\item For the remaining $q-i$ queries $(M_0,M_1)$, answer each of them with $Enc(pk, M_0)$.
\end{itemize}
Note that $\widetilde{LR}_0^{pk}$ perfectly simulates $LR_0^{pk}(pk)$ and $\widetilde{LR}_{q}^{pk}$ perfectly simulates $LR_1^{pk}(pk)$.
Hence
$$\begin{aligned}
& \Adv_{PKE}^{ind-cpa}(A) \\
=& \bigg| \Pr[(pk,sk)\getsr Kg(1^\lambda) : A^{LR_0^{pk}}(pk)=1] 
-\Pr[(pk,sk)\getsr Kg(1^\lambda) : A^{LR_1^{pk}}(pk)=1] \bigg|\\
=& \bigg| \Pr[(pk,sk)\getsr Kg(1^\lambda) : A^{\widetilde{LR}_0^{pk}}(pk)=1] 
-\Pr[(pk,sk)\getsr Kg(1^\lambda) : A^{\widetilde{LR}_q^{pk}}(pk)=1] \bigg|\\
=& \bigg| \sum_{i=1}^q \Delta_i(A) \bigg|.
\end{aligned}$$
\end{proof}

{\bf b) [Points: 6]} For $i \in \{1, \cdots, q\}$, give a PPT adversary $A_i$ such that $\Adv_{PKE}^{ind-cpa}(A_i) = |\Delta_i(A)|$.
\begin{proof}
We give the construction of $A_i$ as following.
\begin{quote}
{\bf Adversary} $A_i^{O}(pk)$: // $O$ can be $LR_0^{pk}$ or $LR_1^{pk}$
\begin{enumerate}
\item $b \gets A^{\tilde{O}}$
\item {\bf return} $b$
\end{enumerate}
\end{quote}
Here the oracle $\tilde{O}$ is defined as:
\begin{itemize}
\item For the first $i-1$ queries $(M_0,M_1)$, answer each of them with $Enc(pk, M_1)$.
\item For the $i$-th query $(M_0,M_1)$, feed it to the oracle $O$ and returns its output as answer.
\item For the remaining $q-i$ queries $(M_0,M_1)$, answer each of them with $Enc(pk, M_0)$.
\end{itemize}
Note that if $O = LR_0^{pk}$, then $\tilde{O}$ simulates $\widetilde{LR}_{i-1}^{pk}$; if $O = LR_1^{pk}$, then $\tilde{O}$ simulates $\widetilde{LR}_{i}^{pk}$. Based on the definition of $\Delta_i(A)$, it immediately follows 
$$\begin{aligned}
& \Adv_{PKE}^{ind-cpa}(A_i) \\
=& \bigg| \Pr[(pk,sk) \getsr Kg(1^\lambda) : A_i^{LR_0^{pk}}(pk)=1] - \Pr[(pk,sk) \getsr Kg(1^\lambda) : A_i^{LR_1^{pk}}(pk)=1] \bigg| \\
=& \bigg| \Pr[(pk,sk) \getsr Kg(1^\lambda) : A^{\widetilde{LR}_{i-1}^{pk}}(pk)=1] - \Pr[(pk,sk) \getsr Kg(1^\lambda) : A_i^{\widetilde{LR}_{i}^{pk}}(pk)=1] \bigg| \\
=& |\Delta_i(A)|.
\end{aligned}$$
\end{proof}

{\bf c) [Points: 4]} Give $A'$ satisfying (\ref{eq:1}), using $A_1, \cdots, A_q$ as subroutines.
\begin{proof}
It can be constructed as following.
\begin{quote}
{\bf Adversary} $A'^O(pk)$: // $O$ can be $LR_0^{pk}$ or $LR_1^{pk}$
\begin{enumerate}
\item $i \getsr \{1, 2, \cdots, q\}$
\item $b \gets A_i^O(pk)$
\item {\bf return} $b$
\end{enumerate}
\end{quote}
\end{proof}

{\bf d) [Points: 2]} How does this simplify the task of proving ind-cpa security of $PKE$? Is an
analogous statement true in the secret-key case?

{\bf Solution}: One query is enough! We only need to consider the games that allow at most one query, i.e., if all the advantages for such games are negligible, then we can immediately conclude that $PKE$ is ind-cpa secure.

It is not true for the secret-key case. Intuitively, for secret-key case, the adversary cannot encrypt any message all by herself, since there is no such analogue of public key $pk$. Thus there is no way to simulate the hybrid oracles in the proof. Any adversary must rely on the encryption oracle to get multiple ciphertexts on chosen messages.

\begin{thebibliography}{10}
\bibitem{BKR94}
Bellare M, Kilian J, Rogaway P. The security of cipher block chaining. In Advances in Cryptology - CRYPTO’94 1994 Aug 21 (pp. 341-358).

\bibitem{Shoup09}
Shoup V. A computational introduction to number theory and algebra. Cambridge university press; 2009.
\end{thebibliography}
\end{document}
