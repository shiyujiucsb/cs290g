\documentclass[12pt]{article}
\usepackage{url,amsmath,setspace,amssymb,amsthm,amsfonts}
%\usepackage{hyperref}

\setlength{\oddsidemargin}{.25in}
\setlength{\evensidemargin}{.25in}
\setlength{\textwidth}{6.25in}
\setlength{\topmargin}{-0.4in}
\setlength{\textheight}{8.5in}

\newcommand{\heading}[5]{
   \renewcommand{\thepage}{#1-\arabic{page}}
   \noindent
   \begin{center}
   \framebox[\textwidth]{
     \begin{minipage}{0.9\textwidth} \onehalfspacing
       {\bf CS 290G -- Introduction to Modern Cryptography} \hfill #2

       {\centering \Large #5
       
       }\medskip

       {\it #3 \hfill #4}
     \end{minipage}
   }
   \end{center}
}

\newcommand{\handout}[3]{\heading{#1}{#2}{Instructor:
Stefano Tessaro}{Student: Shiyu Ji, Yi Yang}{#3}}

\setlength{\parindent}{0in}

\newcommand{\eqdef}{\stackrel{def}{=}}
\newcommand{\N}{\mathbb{N}}
\newcommand{\R}{\mathbb{R}}
\newcommand{\Z}{\mathbb{Z}}
\newcommand{\F}{\mathbb{F}}
\newcommand{\bits}{\{0,1\}}
\newcommand{\inr}{\in_{\mbox{\tiny R}}}
%\newcommand{\getsr}{\gets_{\mbox{\tiny R}}}
\newcommand{\getsr}{\stackrel{\$}{\gets}}
\newcommand{\getsn}{\stackrel{n}{\gets}}
\newcommand{\st}{\mbox{ s.t. }}
\newcommand{\etal}{{\it et al }}
\newcommand{\into}{\rightarrow}

\newcommand{\Ex}{\mathbb{E}}
\newcommand{\e}{\epsilon}
\newcommand{\ee}{\varepsilon}
\newcommand{\ceil}[1]{{\lceil{#1}\rceil}}
\newcommand{\floor}[1]{{\lfloor{#1}\rfloor}}
\newcommand{\angles}[1]{\langle #1 \rangle}
\newcommand{\Com}{{\sf Com}}
\newcommand{\desc}{{\sf desc}}

\newcommand{\rightstep}[1]{%
$\underrightarrow{\quad #1 \quad}$ }

\newcommand{\leftstep}[1]{%
$\underleftarrow{\quad #1 \quad}$ }
\newcommand{\Adv}{\textsf{Adv}}
\newcommand{\tab}{\hspace{0.3in}}
\newcommand{\CBC}{\textsf{CBC}}
\newcommand{\Eval}{\textsf{Eval}}
\newcommand{\Vrfy}{\textsf{Vrfy}}
\newcommand{\WIN}{\textsf{WIN}}
\newcommand{\true}{\textsf{true}}
\newcommand{\false}{\textsf{false}}

%%%%%%%%%%%%%%%%%%%%%%%%%%%%
% Theorems & Definitions


\newtheorem{theorem}{Theorem}[section]

\newtheorem{claim}[theorem]{Claim}
\newtheorem{subclaim}{Claim}[theorem]
\newtheorem{proposition}[theorem]{Proposition}
\newtheorem{lemma}[theorem]{Lemma}
\newtheorem{corollary}[theorem]{Corollary}
\newtheorem{conjecture}[theorem]{Conjecture}
\newtheorem{observation}[theorem]{Observation}

\theoremstyle{definition}
\newtheorem{definition}[theorem]{Definition}
\newtheorem{construction}[theorem]{Construction}
\newtheorem{example}[theorem]{Example}
\newtheorem{counterexample}[theorem]{Counterexample}
\newtheorem{algorithm1}[theorem]{Algorithm}
\newtheorem{protocol}[theorem]{Protocol}
\newtheorem{remark}[theorem]{Remark}
\newtheorem{assumption}[theorem]{Assumption}
\newtheorem{fact}[theorem]{Fact}

%\bibliographystyle{plain}

\begin{document}
\handout{3}{Due: Feb 26, 2016}{Homework 3}
\section{Task 1 - CBC-MAC}
We define a variant of CBC-MAC as the following function $\CBC : \bits^\lambda \times \bits^\lambda \to \bits^n$ using a block cipher $E : \bits^\lambda \times \bits^n \to \bits^n$:
\begin{quote}
{\bf Procedure} $\CBC_k (M)$: // $M \in bits^*$
\begin{enumerate}
\item $M[1], \cdots, M[\ell] \getsn M$
\item $S_0 \gets 0^n$
\item {\bf for} $i=1$ to $\ell$ {\bf do} $S_i \gets E_k (S_{i-1} \oplus M[i])$
\item {\bf return} $S_{\ell}$
\end{enumerate}
\end{quote}
Recall that the $\getsn$ padding operation first appends to $M$ a single 1 bit and then appends as many 0's as necessary to make the length of $M$ a multiple of $n$ (say $\ell \cdot n$). Finally, it outputs the $\ell$ $n$-bit blocks of the extension of $M$.

{\bf a) [Points: 8]} Prove that the $\CBC$ construction given above is \emph{not} UF-CMA secure. In particular, show first that given $Y = \CBC_k (M)$ for some message $M$ and (unknown) key $k$, there exist two other messages $M', M''$ s.t. $M'' \not\in \{M, M'\}$ with the following property:
\begin{quote}
Using $Y$ and the result $Y' = \CBC_k (M')$ of the query $M'$ to $\CBC_k$, we can compute $Y'' = \CBC_k (M'')$ without querying $M$.
\end{quote}
Use this to describe a concrete attacker $A$ making two queries!

\begin{proof}
We directly give the adversary, and then observe the $M$, $M'$, $Y$ and $Y'$ that are used.
\begin{quote}
{\bf Adversary} $A^{\Eval, \Vrfy}$:
\begin{enumerate}
\item $\WIN \gets \false$
\item $M \gets 0^n$
\item $Y \gets \Eval(M)$
\item $M' \gets 1^n$
\item $Y' \gets \Eval(M')$
\item Query $\Vrfy(0^n10^{n-1}||(Y \oplus 1^n), Y')$
\end{enumerate}
\end{quote}
Note that 
$$Y = \CBC_k (0^n) = E_k(E_k(0^n) \oplus 10^{n-1}),$$
$$Y' = \CBC_k (1^n) = E_k(E_k(1^n) \oplus 10^{n-1}).$$
Hence
$$\begin{aligned}
Y' &= E_k(E_k(Y \oplus Y \oplus 1^n) \oplus 10^{n-1}) \\
&= E_k(E_k(E_k(E_k(0^n) \oplus 10^{n-1}) \oplus Y \oplus 1^n) \oplus 10^{n-1}) \\
&= \CBC_k (0^n10^{n-1} || (Y \oplus 1^n) || 10^{n-1}).
\end{aligned}$$
Thus $\WIN$ will always be set to $\true$ in the last $\Vrfy$. Hence we have $\Adv_{CBC}^{uf-cma}(A) = 1$.
\end{proof}
Note that the adversary $A$ can also query $\Vrfy(0^n10^{n-1}||Y, Y)$. The advantage is still 1. Hence only one query $\Eval(0^n)$ would be enough.

{\bf b) [Points: 4]} How can we modify the encoding of $M$ into blocks
$M[1], \cdots, M[\ell]$ so that the attack from {\bf a)} is not possible any more?

{\bf Solution}: Note that in {\bf a)} we query two messages: $0^n$ and $0^n10^{n-1}||(Y \oplus 1^n)$. $0^n$ is a proper prefix of the other. If we require any queried message cannot be a prefix of any other, such attack can be defended. As suggested by \cite{BKR94}, this can be done by prepending the block length $\ell$ before $M[1], \cdots, M[\ell]$.
\begin{quote}
{\bf Procedure} $\CBC_k (M)$: // $M \in bits^*$
\begin{enumerate}
\item $M[1], \cdots, M[\ell] \getsn M$
\item $M[0] \gets \ell$
\item $S_{-1} \gets 0^n$
\item {\bf for} $i=0$ to $\ell$ {\bf do} $S_i \gets E_k (S_{i-1} \oplus M[i])$
\item {\bf return} $S_{\ell}$
\end{enumerate}
\end{quote}
There may be some other ways to achieve this goal.

\begin{thebibliography}{10}
\bibitem{BKR94}
Bellare M, Kilian J, Rogaway P. The security of cipher block chaining. In Advances in Cryptology - CRYPTO’94 1994 Aug 21 (pp. 341-358).
\end{thebibliography}
\end{document}
