\documentclass[12pt]{article}
\usepackage{url,amsmath,setspace,amssymb,amsthm,amsfonts}
%\usepackage{hyperref}


\setlength{\oddsidemargin}{.25in}
\setlength{\evensidemargin}{.25in}
\setlength{\textwidth}{6.25in}
\setlength{\topmargin}{-0.4in}
\setlength{\textheight}{8.5in}

\newcommand{\heading}[5]{
   \renewcommand{\thepage}{#1-\arabic{page}}
   \noindent
   \begin{center}
   \framebox[\textwidth]{
     \begin{minipage}{0.9\textwidth} \onehalfspacing
       {\bf CS 290G -- Introduction to Modern Cryptography} \hfill #2

       {\centering \Large #5
       
       }\medskip

       {\it #3 \hfill #4}
     \end{minipage}
   }
   \end{center}
}

\newcommand{\handout}[3]{\heading{#1}{#2}{Instructor:
Stefano Tessaro}{Student: Shiyu Ji, Yi Yang}{A Tutorial on the Notions of Obfuscations}}

\setlength{\parindent}{0in}

\newcommand{\eqdef}{\stackrel{def}{=}}
\newcommand{\N}{\mathbb{N}}
\newcommand{\R}{\mathbb{R}}
\newcommand{\Z}{\mathbb{Z}}
\newcommand{\F}{\mathbb{F}}
\newcommand{\bits}{\{0,1\}}
\newcommand{\inr}{\in_{\mbox{\tiny R}}}
%\newcommand{\getsr}{\gets_{\mbox{\tiny R}}}
\newcommand{\getsr}{\stackrel{\$}{\gets}}
\newcommand{\st}{\mbox{ s.t. }}
\newcommand{\etal}{{\it et al }}
\newcommand{\into}{\rightarrow}

\newcommand{\Ex}{\mathbb{E}}
\newcommand{\e}{\epsilon}
\newcommand{\ee}{\varepsilon}
\newcommand{\ceil}[1]{{\lceil{#1}\rceil}}
\newcommand{\floor}[1]{{\lfloor{#1}\rfloor}}
\newcommand{\angles}[1]{\langle #1 \rangle}
\newcommand{\Com}{{\sf Com}}
\newcommand{\desc}{{\sf desc}}

\newcommand{\rightstep}[1]{%
$\underrightarrow{\quad #1 \quad}$ }

\newcommand{\leftstep}[1]{%
$\underleftarrow{\quad #1 \quad}$ }

\newcommand{\Adv}{\mathsf{Adv}}
\newcommand{\poly}{\mathsf{poly}}
\newcommand{\MAC}{\mathsf{MAC}}
\newcommand{\Ver}{\mathsf{Ver}}

\newcommand{\tab}{\hspace{0.3in}}

\newcommand{\io}{$i\mathcal{O}$}
\newcommand{\dio}{$di\mathcal{O}$}
%%%%%%%%%%%%%%%%%%%%%%%%%%%%
% Theorems & Definitions


\newtheorem{theorem}{Theorem}[section]

\newtheorem{claim}[theorem]{Claim}
\newtheorem{subclaim}{Claim}[theorem]
\newtheorem{proposition}[theorem]{Proposition}
\newtheorem{lemma}[theorem]{Lemma}
\newtheorem{corollary}[theorem]{Corollary}
\newtheorem{conjecture}[theorem]{Conjecture}

\theoremstyle{definition}
\newtheorem{definition}[theorem]{Definition}
\newtheorem{construction}[theorem]{Construction}
\newtheorem{example}[theorem]{Example}
\newtheorem{algorithm1}[theorem]{Algorithm}
\newtheorem{protocol}[theorem]{Protocol}
\newtheorem{remark}[theorem]{Remark}
\newtheorem{observation}[theorem]{Observation}
\newtheorem{assumption}[theorem]{Assumption}
\newtheorem{fact}[theorem]{Fact}

%\bibliographystyle{plain}
\usepackage{tikz}
\usetikzlibrary{calc,decorations.pathreplacing}

\begin{document}
%\handout{}{\today{}}{}
\title{A Tutorial on the Notions of Obfuscations}
\author{Shiyu Ji, Yi Yang}
\date{\today}
\maketitle

\section{Introduction}
Cryptographic obfuscation has received extensive attention from the research community this century. Informally, obfuscation is a family of algorithms (or obfuscators) that can transform the input (e.g., programs or circuits) to some ``unintelligible'' form, while the functionality of the input is preserved. For example, a good obfuscator is like a compiler, which takes a program written in source code as input, and outputs quite unreadable binary code. Ideally, the binary code should not reflect any information about the source code except the functionality, i.e., input-output behavior. Given only the binary code, any adversary cannot learn any information except that can be learned by observing the inputs and outputs of the program, i.e., observing the black box of the program. This ideal and natural notion can completely defeat any attempts to reveal the source code given the obfuscated code. Such attempts include software reverse engineering. Hence it would be a great news if this notion can be achieved. Unfortunately, Barak et al. \cite{barak2001possibility} have shown that there is no \emph{general} way to achieve this notion, which was formalized as \emph{Virtual Black Box} (VBB) in their paper. Even though there is still some hope to achieve VBB for specific inputs, e.g., point functions \cite{canetti1997towards, canetti1998perfectly, wee2005obfuscating}, the notion of VBB is generally too strong in the sense that it is likely that most programs of interest cannot be VBB obfuscated in practice \cite{goldwasser2005impossibility}. Hence after \cite{barak2001possibility}, the possibilities of relaxation have also been extensively investigated.

Due to the pessimistic impossibility results of VBB, Barak et al. \cite{barak2001possibility} suggested two relaxed notions of obfuscation: \emph{Indistinguishability Obfuscation} (\io) and further relaxed \emph{Differing-input Obfuscation} (\dio).
\bibliographystyle{alpha}
\bibliography{obfs}
	
\end{document}
